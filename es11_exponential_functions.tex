\documentclass[11pt]{article}
%%%%%%%%%%PACKAGES%%%%%%%%%%%%%%%%%%%%%%%%%%%%%%%%%%%
\usepackage{latexsym}
\usepackage{amssymb, amsmath, amsthm, amsfonts}
\usepackage{stmaryrd} %For \mapsfrom
%\usepackage[fleqn]{amsmath}  % fleqn option makes aligned equations flushed left!
%\usepackage[english]{babel}
%\usepackage{pgf}
\usepackage{mathtools}
\usepackage[mathscr]{eucal}
\usepackage{fancyhdr}
\usepackage{multicol,parcolumns}
\usepackage{enumerate}
%\usepackage{enumitem}
\usepackage[shortlabels]{enumitem}
\usepackage{graphicx}
\usepackage{extarrows}
\usepackage{cancel}
%\usepackage{tikz}
%\usepackage[all,cmtip]{xy} %\SelectTips{cm}{10}
\usepackage[all]{xy} \SelectTips{cm}{10}
%\usepackage{listings} %For code blocks

\input{LatexPreamble}
%%%%%%%%FANCY HEADER%%%%%%%%%
\pagestyle{plain}
\setlength{\headheight}{13.6pt}
\fancyhfoffset[L]{.5in}
%\lhead{\Large \bf{Name:}}
\chead{Executive summary: exponential functions}
\rhead{Math 220-2}
%\lfoot{TURN OVER!}
%\rfoot{TURN OVER!}

%%%%%%%PAGE LAYOUT%%%%%%%%%%%%%
\setlength{\textwidth}{6.5in}
\setlength{\textheight}{9in}

%\setlength{\topmargin}{-.8in}
%\setlength{\columnsep}{1.5in}
\addtolength{\hoffset}{-1 in}
\addtolength{\voffset}{-.5 in}


%%%%%%%THEOREM ENVIRONMENTS%%%%%%%%
\theoremstyle{definition}
\newtheorem*{definition}{Definition}
\newtheorem*{definitions}{Definitions}
\newtheorem*{notation}{Notation}
\newtheorem*{example}{Example}
\newtheorem*{comment}{Comment}
\newtheorem*{comments}{Comments}
\newtheorem*{examples}{Examples}
\newtheorem*{warning}{Warning}
\newtheorem*{theorem}{Theorem}
\newtheorem*{corollary}{Corollary}
\newtheorem*{proposition}{Proposition}
\newtheorem*{lemma}{Lemma}

\newtheoremstyle{named}{}{}{}{}{\bfseries}{.}{.5em}{\thmnote{#3}}
\theoremstyle{named}
\newtheorem*{namedtheorem}{Theorem}

\newcounter{myalgctr}
\newenvironment{myalg}{%      define a custom environment
   \bigskip\noindent%         create a vertical offset to previous material
   \refstepcounter{myalgctr}% increment the environment's counter
   \textbf{Algorithm \themyalgctr}% or \textbf, \textit, ...
   \newline%
   }{\par\bigskip}  %
\numberwithin{myalgctr}{section}



\newenvironment{solution}{\begin{proof}[Solution]}{\end{proof}}


%%%%%%%%%%HYPERREFS PACKAGE%%%%%%%%%%%%%%%%%
\usepackage[colorlinks]{hyperref}
%\definecolor{webcolor}{rgb}{0.8,0,0.2}
%\definecolor{webbrown}{rgb}{.6,0,0}
%\usepackage[
%        colorlinks,
%       linkcolor=webbrown,  filecolor=webcolor,  citecolor=webbrown,
%        backref
%]{hyperref}
\usepackage[alphabetic, lite]{amsrefs} % for bibliography
\begin{document}
\thispagestyle{fancy}
\subsection*{Definitions}
\begin{namedtheorem}[Exponential function] The {\bf exponential function}, denoted $\exp$, is defined as the inverse of the natural logarithm function. In other words, letting $f(x)=\ln x$, we have $f^{-1}(x)=\exp(x)$. We also write $e^x$ for $\exp(x)$.
\end{namedtheorem}
% \begin{comment}
% By definition $e^x=\exp(x)$ is the inverse function of $f(x)=\ln x$. You might wonder how this is consistent with our usual notion of $e^x$ as ``raising $e$ to the $x$-th power". Let's examine this question in stages, letting $x$ be an increasiningly complicated type of input.
% \\
% Case: $x=n$ is a postive integer. Then by definition
% \[
%  \exp(n)=\exp(n\ln e)=\exp(\ln(e^n) )=e^n=\underset{\text{$n$ times}}{\underbrace{e\cdot e\cdots e}}.
% \]
% Case: $x=-n$ is a negative integer. Then by definition
% \[
%  \exp(-n)=\exp(-n\ln e)=\exp(\ln(e^{-n}) )=e^{-n}=\frac{1}{e^n}.
% \]
% Case: $x=m/n$ is a rational number. Then by definition
% \[
%  \exp(m/n)=\exp((m/n)\ln e)=\exp(\ln(e^{m/n}) )=e^{m/n}=\sqrt[n]{e^m}.
% \]
% Case: $x$ is irrational (i.e., not a rational number). Previously, we had no way of defining ``$e$ to the power $x$", but now we do: namely, $e^x=\exp(x)$ is the positive number $y$ satisfying $\ln y=x$.
% \end{comment}
\begin{namedtheorem}[Exponential function with base $a$] Let $a$ be a fixed positive number. The {\bf exponential function with base $a$}, denoted $f(x)=a^x$, is the function with domain all real numbers defined as
  \[
  a^x=e^{x\ln a}.
  \]

\end{namedtheorem}
\begin{namedtheorem}[Logarithmic function with base $a$] Let $a$ be a fixed positive number. The {\bf logarathmic function with base $a$}, denoted $f(x)=\log_a(x)$ is defined as the inverse function of $g(x)=a^x$.

\end{namedtheorem}

%***********************************************
 \subsection*{Theory}
\begin{namedtheorem}[Properties of the exponential function]
  The following properties hold:
  \begin{enumerate}[itemsep=0pt,topsep=0pt]
    \item The exponential function is differentiable (hence also continuous) on all of $\mathbb{R}$ and satisfies
    \[
    \frac{d}{dx}e^x=e^x.
    \]
    Equivalently, we have
    \[
    \int e^x\, dx=e^x+C.
    \]
    \item The exponential function is increasing and hence one-to-one. Its graph is always concave up.
    \item We have
    \begin{align*}
      \lim_{x\to\infty}e^x&=\infty\\
      \lim_{x\to -\infty}e^x&=0
    \end{align*}
    \item The domain of $\exp$ is $\mathbb{R}=(-\infty, \infty)$; the range of $\exp$ is $(0,\infty)$.
    \item $e^0=1$.
    \item We have
    \begin{align*}
      e^{x+y}&=e^xe^y &
      e^{x-y}&=e^x/e^y &
      e^{xy}&=(e^x)^y
    \end{align*}
    for all $x,y\in\mathbb{R}$.
    \item We have
    \begin{align*}
      \ln(e^x)&=x, \text{ for all $x$}; &
      e^{\ln x}&=x, \text{ for all $x\in (0,\infty)$.}
    \end{align*}
  \end{enumerate}
\end{namedtheorem}
\begin{namedtheorem}[Logarithmic and exponential compendium] The table below summarizes the important properties of our various families of logarithmic and exponential functions.
\[
\begin{array}{c|c|c|c||c|c|c|}
  f(x) & \ln x& \log_a x,\, a>1& \log_a x,\, 0<a<1&  e^x & a^x,\, a>1 & a^x,\, 0<a<1 \\
  \hline
  \text{Domain} &\multicolumn{3}{|c||}{(0,\infty)}&\multicolumn{3}{c|}{(-\infty, \infty)}\\
  \hline
  \text{Range} &\multicolumn{3}{|c||}{(-\infty,\infty)}&\multicolumn{3}{c|}{(0, \infty)}\\
  \hline
  \text{Monotonicity}&\multicolumn{2}{|c|}{\text{Increasing}}&\text{Decreasing}&\multicolumn{2}{|c|}{\text{Increasing}}&\text{Decreasing}\\
  \hline
  \text{Limit as }x\to\infty& \multicolumn{2}{|c|}{\infty}&-\infty &  \multicolumn{2}{|c|}{\infty}& 0\\
  \hline
  \text{Limit as }x\to 0^+ & \multicolumn{2}{|c|}{-\infty}& \infty & \multicolumn{3}{|c|}{*}\\
  \hline
  \text{Limit as }x\to-\infty & \multicolumn{3}{|c||}{*}&  \multicolumn{2}{|c|}{0}& \infty\\
  \hline
  \text{Inverse}&e^x&\multicolumn{2}{|c||}{a^x}& \ln x& \multicolumn{2}{|c|}{\log_a x}\\
  \hline
  \text{Relation to base-$e$}&\ln x=\log_e x&\multicolumn{2}{|c||}{\log_a x=\frac{\ln x}{\ln a}} &* &\multicolumn{2}{|c|}{a^x=e^{x\ln a}}\\
  \hline
  \text{Algebra}&\multicolumn{3}{|c||}{\begin{array}{c}
    \log_a(xy)=\log_ax+\log_ay\\
    \log_a(x^y)=y\log_a x\\
    \log_a(a^x)=x
  \end{array}
  }
  &
  \multicolumn{3}{|c||}{\begin{array}{c}
    a^{x+y}=a^xa^y\\
    a^{xy}=(a^x)^y\\
    a^{\log_a x}=x
  \end{array}
  }\\
  \hline
\end{array}
\]


\end{namedtheorem}

\begin{namedtheorem}[Derivative/antiderivative compendium] We collect here the new derivative formulas obtained via logarithms and exponential functions, along with their equivalent antiderivative formulas.
\begin{align*}
  \frac{d}{dx}\ln\vert x\vert=\frac{1}{x} & \iff \int\frac{1}{x} \, dx=\ln\vert x\vert+C \\
  \frac{d}{dx}\ln\vert \cos x \vert=-\tan x & \iff \int\tan x \, dx=-\ln\vert \cos x \vert+C=\ln\vert\sec x\vert+C \\
  \frac{d}{dx}\ln\vert \sin x\vert=\cot x & \iff \int\cot x \, dx=\ln\vert \sin x\vert+C \\
  \frac{d}{dx}\ln\vert \sec x+\tan x\vert=\sec x & \iff \int\sec x \, dx=\ln\vert \sec x+\tan x\vert+C \\
  \frac{d}{dx}\ln\vert \csc x+\cot x\vert=-\csc x & \iff \int\csc x \, dx=-\ln\vert \csc x+\cot x\vert+C\\
  \frac{d}{dx}e^x=e^x & \iff \int e^x \, dx=e^x+C\\
  \frac{d}{dx}a^x=(\ln a)a^x & \iff \int a^x \, dx=\frac{1}{\ln a}a^x+C\\
  \frac{d}{dx}\log_a \vert x\vert =\frac{1}{(\ln a)\, x} & \iff \int \frac{1}{(\ln a)\, x} \, dx=\log_a\vert x\vert+C
\end{align*}

\end{namedtheorem}


%***************************************

% \subsection*{Procedures}

%*********************************************************
\subsection*{Examples}
\begin{enumerate}
  \item Find all $t$ satisfying $\displaystyle 2^{-t^2}=\frac{1}{16}$. Simplify your answer as much as possible.  
  \item Compute $f'(x)$ for each of the following functions.
  \begin{enumerate}
    \item $f(x)=\ln(\sin x)e^{\cos x}$
    \item $f(x)=\log_3(2^x+3^{x^2})$
  \end{enumerate}
  \item Compute the following definite indefinite integrals.
  \begin{enumerate}
    \item $\displaystyle\int (e^t)^2\sin(e^{2t})\, dt$
    \item $\displaystyle\int_0^\pi \sin(2^x)2^{\cos(2^x)+x}\, dx$
  \end{enumerate}
\end{enumerate}




\end{document}
