\documentclass[11pt]{article}
%%%%%%%%%%PACKAGES%%%%%%%%%%%%%%%%%%%%%%%%%%%%%%%%%%%
\usepackage{latexsym}
\usepackage{amssymb, amsmath, amsthm, amsfonts}
\usepackage{stmaryrd} %For \mapsfrom
%\usepackage[fleqn]{amsmath}  % fleqn option makes aligned equations flushed left!
%\usepackage[english]{babel}
%\usepackage{pgf}
\usepackage{mathtools}
\usepackage[mathscr]{eucal}
\usepackage{fancyhdr}
\usepackage{multicol,parcolumns}
\usepackage{enumerate}
%\usepackage{enumitem}
\usepackage[shortlabels]{enumitem}
\usepackage{graphicx}
\usepackage{extarrows}
\usepackage{cancel}
%\usepackage{tikz}
%\usepackage[all,cmtip]{xy} %\SelectTips{cm}{10}
\usepackage[all]{xy} \SelectTips{cm}{10}
%\usepackage{listings} %For code blocks

\input{LatexPreamble}
%%%%%%%%FANCY HEADER%%%%%%%%%
\pagestyle{plain}
\setlength{\headheight}{13.6pt}
\fancyhfoffset[L]{.5in}
%\lhead{\Large \bf{Name:}}
\chead{Executive summary: trigonometric substitution}
\rhead{Math 220-2}
%\lfoot{TURN OVER!}
%\rfoot{TURN OVER!}

%%%%%%%PAGE LAYOUT%%%%%%%%%%%%%
\setlength{\textwidth}{6.5in}
\setlength{\textheight}{9in}

%\setlength{\topmargin}{-.8in}
%\setlength{\columnsep}{1.5in}
\addtolength{\hoffset}{-1 in}
\addtolength{\voffset}{-.5 in}


%%%%%%%THEOREM ENVIRONMENTS%%%%%%%%
\theoremstyle{definition}
\newtheorem*{definition}{Definition}
\newtheorem*{definitions}{Definitions}
\newtheorem*{notation}{Notation}
\newtheorem*{example}{Example}
\newtheorem*{comment}{Comment}
\newtheorem*{comments}{Comments}
\newtheorem*{examples}{Examples}
\newtheorem*{warning}{Warning}
\newtheorem*{theorem}{Theorem}
\newtheorem*{corollary}{Corollary}
\newtheorem*{proposition}{Proposition}
\newtheorem*{lemma}{Lemma}

\newtheoremstyle{named}{}{}{}{}{\bfseries}{.}{.5em}{\thmnote{#3}}
\theoremstyle{named}
\newtheorem*{namedtheorem}{Theorem}

\newcounter{myalgctr}
\newenvironment{myalg}{%      define a custom environment
   \bigskip\noindent%         create a vertical offset to previous material
   \refstepcounter{myalgctr}% increment the environment's counter
   \textbf{Algorithm \themyalgctr}% or \textbf, \textit, ...
   \newline%
   }{\par\bigskip}  %
\numberwithin{myalgctr}{section}



\newenvironment{solution}{\begin{proof}[Solution]}{\end{proof}}


%%%%%%%%%%HYPERREFS PACKAGE%%%%%%%%%%%%%%%%%
\usepackage[colorlinks]{hyperref}
%\definecolor{webcolor}{rgb}{0.8,0,0.2}
%\definecolor{webbrown}{rgb}{.6,0,0}
%\usepackage[
%        colorlinks,
%       linkcolor=webbrown,  filecolor=webcolor,  citecolor=webbrown,
%        backref
%]{hyperref}
\usepackage[alphabetic, lite]{amsrefs} % for bibliography
\begin{document}
\thispagestyle{fancy}
%\subsection*{Definitions}


%***********************************************
%  \subsection*{Theory}
% \begin{namedtheorem}[Reverse substitution] Let $g$ be a one-to-one, differentiable function from the interval $J$ to the interval $I$ satisfying $g'(t)\ne 0$ for all $t$, and suppose $f$ is continuous on $I$.
% \\
% If $F(t)$ is an antiderivative of $f(g(t))g'(t)$ on $J$, then $F(g^{-1}(x))$ is an antiderivative of $f(x)$ on $I$. In other words, setting $x=g(t)$ and $t=g^{-1}(x)$, we have
%   \[
%   \int f(x)\, dx =F(t)+C=F(g^{-1}(x))+C
%   \]
% \end{namedtheorem}


%***************************************

\subsection*{Procedures}
\begin{namedtheorem}[Reverse substitution technique (indefinite integral)] To compute $\displaystyle\int f(x)\, dx$ using reverse substitution, proceed as follows:
\begin{enumerate}[topsep=0pt, itemsep=0pt]
  \item Choose a 1-1, differentiable substitution function $g$ with differentiable inverse and assemble the two equations
  \begin{align*}
    x&=g(t)\\
    dx&=g'(t)\, dt
  \end{align*}
  \item Compute
  \[
  \int f(g(t))g'(t)\, dt=F(t)+C.
  \]
  \item We conclude that
  \[
  \int f(x)\, dx=F(g^{-1}(x))+C.
  \]
  Alternatively, we compute an antiderivative for $f(x)$ by expressing the function $F(t)$ from (2) as a function of $x$ using $x=g(t)$ and $g^{-1}(x)=t$.
\end{enumerate}

\end{namedtheorem}
\begin{samepage}
\begin{namedtheorem}[Reverse substitution technique (definite integral)] To compute $\displaystyle\int_a^b f(x)\, dx$ using reverse substitution substitution, proceed as follows:
\begin{enumerate}[topsep=0pt, itemsep=0pt]
  \item Choose a 1-1, differentiable substitution function $g$ with differentiable inverse and assemble the two equations
  \begin{align*}
    x&=g(t)\\
    dx&=g'(t)\, dt
  \end{align*}
  \item
  Then we have
  \[
  \displaystyle\int_{x=a}^{x=b} f(x)\, dx=\int_{t=g^{-1}(a)}^{t=g^{-1}(b)}f(g(t))g'(t)\, dt.
  \]
\end{enumerate}

\end{namedtheorem}
\end{samepage}
\begin{comment}
What is the difference between our original (forward) substitution and reverse subsitution?
\begin{itemize}
  \item Forward substitution allows us to find an antiderivative of $f(u(x))u'(x)$ from an antiderivative of $f(x)$: namely,
  \[
  F(x) \text{ is an antiderivative of } f(x)\implies F(u(x)) \text{ is an antiderivative  of } f(u(x))u'(x).
  \]
  \item Reverse substitution allows us to find an antiderivative of $f(x)$ from an antiderivative of $f(g(t))g'(t)$: namely,
  \[
  F(t) \text{ is an antiderivative of } f(g(t))g'(t)\implies F(g^{-1}(x)) \text{ is an antiderivative  of } f(x).
  \]
\end{itemize}
\end{comment}
\begin{namedtheorem}[Trigonometric substitution]
The table below indicates potentially helpful (reverse) substitutions for functions $f$ containing particular forms of expressions.
\begin{align*}
  f(x) \text{ contains } \sqrt{a^2-x^2}&\implies \text{try } \begin{array}{c}
    x=a\sin\theta\\
    dx=a\cos\theta\, d\theta
  \end{array}, -\pi/2\leq\theta\leq\pi/2\\
  f(x) \text{ contains } x^2+a^2 &\implies \text{try } \begin{array}{c}
    x=a\tan\theta\\
    dx=a\sec^2\theta\, d\theta
  \end{array}, -\pi/2<\theta< \pi/2\\
  f(x) \text{ contains } \sqrt{x^2-a^2} &\implies \text{try } \begin{array}{c}
    x=a\sec\theta\\
    dx=a\sec\theta\tan\theta\, d\theta
  \end{array}, 0<\theta< \pi/2 \text{ or } \pi/2<\theta<\pi
\end{align*}

\end{namedtheorem}
%*********************************************************
\subsection*{Examples}

\begin{enumerate}
  \item Derive the area formula for a circle of radius $r$ using calculus.
  \item Find an antiderivative of $\sqrt{1-x^2}$.
  \item Compute the following integrals.
  \begin{enumerate}
    \item $\displaystyle\int_{-\sqrt{2}}^{-2/\sqrt{3}}\frac{\sqrt{x^2-1}}{x}\, dx$
    \item $\displaystyle\int \frac{1}{x^2\sqrt{x^2+4}}$
    \item $\displaystyle\int \frac{\sqrt{x^2-1}}{x}\, dx$, $x\leq -1$.

    {\bf Note}. This is the indefinite integral version of (a). To finish the computation you need to use the $\operatorname{arcsec}$ function, which is defined as the inverse of $\sec$ with restricted domain $[-1,\pi/2)\cup (\pi/2,1]$. We don't officially cover $\operatorname{arcsec}$ in this course, but this exercise is good practice nonetheless.
  \end{enumerate}
\end{enumerate}



\end{document}
