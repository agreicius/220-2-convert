\documentclass[11pt]{article}
%%%%%%%%%%PACKAGES%%%%%%%%%%%%%%%%%%%%%%%%%%%%%%%%%%%
\usepackage{latexsym}
\usepackage{amssymb, amsmath, amsthm, amsfonts}
\usepackage{stmaryrd} %For \mapsfrom
%\usepackage[fleqn]{amsmath}  % fleqn option makes aligned equations flushed left!
%\usepackage[english]{babel}
%\usepackage{pgf}
\usepackage{mathtools}
\usepackage[mathscr]{eucal}
\usepackage{fancyhdr}
\usepackage{multicol,parcolumns}
\usepackage{enumerate}
%\usepackage{enumitem}
\usepackage[shortlabels]{enumitem}
\usepackage{graphicx}
\usepackage{extarrows}
\usepackage{cancel}
\usepackage{tikz}
\usepackage{tikz-cd}
\usepackage[all,cmtip]{xy} %\SelectTips{cm}{10}
\usepackage[all]{xy} \SelectTips{cm}{10}
%\usepackage{listings} %For code blocks

\input{LatexPreamble}
%%%%%%%%FANCY HEADER%%%%%%%%%
\pagestyle{plain}
\setlength{\headheight}{13.6pt}
\fancyhfoffset[L]{.5in}
%\lhead{\Large \bf{Name:}}
\chead{Executive summary: integration by parts}
\rhead{Math 220-2}
%\lfoot{TURN OVER!}
%\rfoot{TURN OVER!}

%%%%%%%PAGE LAYOUT%%%%%%%%%%%%%
\setlength{\textwidth}{6.5in}
\setlength{\textheight}{9in}

%\setlength{\topmargin}{-.8in}
%\setlength{\columnsep}{1.5in}
\addtolength{\hoffset}{-1 in}
\addtolength{\voffset}{-.5 in}


%%%%%%%THEOREM ENVIRONMENTS%%%%%%%%
\theoremstyle{definition}
\newtheorem*{definition}{Definition}
\newtheorem*{definitions}{Definitions}
\newtheorem*{notation}{Notation}
\newtheorem*{example}{Example}
\newtheorem*{comment}{Comment}
\newtheorem*{comments}{Comments}
\newtheorem*{examples}{Examples}
\newtheorem*{warning}{Warning}
\newtheorem*{theorem}{Theorem}
\newtheorem*{corollary}{Corollary}
\newtheorem*{proposition}{Proposition}
\newtheorem*{lemma}{Lemma}

\newtheoremstyle{named}{}{}{}{}{\bfseries}{.}{.5em}{\thmnote{#3}}
\theoremstyle{named}
\newtheorem*{namedtheorem}{Theorem}

\newcounter{myalgctr}
\newenvironment{myalg}{%      define a custom environment
   \bigskip\noindent%         create a vertical offset to previous material
   \refstepcounter{myalgctr}% increment the environment's counter
   \textbf{Algorithm \themyalgctr}% or \textbf, \textit, ...
   \newline%
   }{\par\bigskip}  %
\numberwithin{myalgctr}{section}



\newenvironment{solution}{\begin{proof}[Solution]}{\end{proof}}


%%%%%%%%%%HYPERREFS PACKAGE%%%%%%%%%%%%%%%%%
\usepackage[colorlinks]{hyperref}
%\definecolor{webcolor}{rgb}{0.8,0,0.2}
%\definecolor{webbrown}{rgb}{.6,0,0}
%\usepackage[
%        colorlinks,
%       linkcolor=webbrown,  filecolor=webcolor,  citecolor=webbrown,
%        backref
%]{hyperref}
\usepackage[alphabetic, lite]{amsrefs} % for bibliography
\begin{document}
\thispagestyle{fancy}
\subsection*{Definitions}


%***********************************************
 \subsection*{Theory}

\begin{namedtheorem}[Integration by parts rule]
Let $u$ and $v$ be continuously differentiable functions on an interval $I$ containing the interval $[a,b]$.
\begin{enumerate}
  \item {\em Indefinite integral form}. We have
  \[
  \int u(x)v'(x)\, dx= u(x)v(x)-\int u'(x)v(x)\, dx.
  \]
  \item {\em Definite integral form}. We have
  \[
  \int_a^b u(x)v'(x)\, dx= u(x)v(x)\Bigr]_a^b-\int_a^b u'(x)v(x)\, dx.
  \]
\end{enumerate}
\end{namedtheorem}

%***************************************

\subsection*{Procedures}
\begin{namedtheorem}[The art of by parts]
To use the integration by parts technique on an integral of the form $\displaystyle\int f(x)g(x)\, dx$ proceed as follows:
\begin{enumerate}
  \item {\em Who is $u$, and who $v'$}? Declare one of $f$ and $g$ to be $u$ and the other to be $v'$. The mnemonic device LIPET ((L)og, (I)nverse trig, (P)olynomial/radical, (E)xponent, (T)rig) often leads to a useful choice of $u$.
  \item {\em Assemble ingredients}. Suppose without loss of generality that we have chosen $u=f$ and $v'=g$. Then compute the derivative $f'$ of $f$ and compute an {\em antiderivative} $G$ of $g$:
  \[
  \begin{tikzcd}
    u(x)=f(x) \ar[d,bend right=80,"\text{compute derivative}"'] & v'(x)=g(x) \ar[d,bend left=80,"\text{compute antiderivative}"]\\
    u'(x)=f'(x) & v(x)=G(x)
  \end{tikzcd}
  \]
  \item Apply the integration by parts rule with ingredients assembled in (2):
  \begin{align*}
    \int \underset{u}{f(x)}\underset{v'}{g(x)}\, dx&= \underset{u}{f(x)}\underset{v}{G(x)}-\int \underset{u'}{f'(x)}\underset{v}{G(x)}\, dx.
  \end{align*}
  \end{enumerate}

\end{namedtheorem}
\begin{namedtheorem}[Integration workflow]
For many integral computations it will be clear whether to use a formula, substitution, or integration by parts. When it is not clear how to proceed, the following {\em rough} workflow might be helpful.
\begin{enumerate}[itemsep=0pt, topsep=0pt]
  \item {\em Formula}. If possible, use an integration formula, perhaps after some simple algebraic preparation. Otherwise, move to (2).
  \item {\em Substitution}. Evaluate whether a substitution could transform the integral into one where (1) applies. If not promising, move to (3).
  \item {\em By parts}. Evaluate whether the integral is amenable to a by parts approach. You may want to mentally run through a couple of choices of ``who is $u$, and who $v'$". If not promising, move to (4).
  \item {\em Algebraic techniques}. Consider more creative algebraic techniques, including trigonometric identities. If applicable, return to (1).
\end{enumerate}

\end{namedtheorem}

%*********************************************************
\subsection*{Examples}
Compute the following integrals using integration by parts. (You might explore whether the integral could also be computed using substitution.)
\begin{enumerate}
  \item Compute $\ds\int_0^1 xe^{-x}\, dx$
  \item Compute $\ds\int x^2e^x\, dx$
  \item Compute $\ds\int \ln \vert x\vert\, dx$
  \item Compute $\ds\int \frac{x^3}{x^2+1}\, dx$
  \item Compute $\ds\int \arctan x\, dx$
  \item Compute $\int e^x\cos x\, dx$
\end{enumerate}




\end{document}
