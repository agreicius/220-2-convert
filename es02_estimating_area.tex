\documentclass[11pt]{article}
%%%%%%%%%%PACKAGES%%%%%%%%%%%%%%%%%%%%%%%%%%%%%%%%%%%
\usepackage{latexsym}
\usepackage{amssymb, amsmath, amsthm, amsfonts}
\usepackage{stmaryrd} %For \mapsfrom
%\usepackage[fleqn]{amsmath}  % fleqn option makes aligned equations flushed left!
%\usepackage[english]{babel}
%\usepackage{pgf}
\usepackage{mathtools}
\usepackage[mathscr]{eucal}
\usepackage{fancyhdr}
\usepackage{multicol,parcolumns}
\usepackage{enumerate}
%\usepackage{enumitem}
\usepackage[shortlabels]{enumitem}
\usepackage{graphicx}
\usepackage{extarrows}
\usepackage{cancel}
%\usepackage{tikz}
%\usepackage[all,cmtip]{xy} %\SelectTips{cm}{10}
\usepackage[all]{xy} \SelectTips{cm}{10}
%\usepackage{listings} %For code blocks

\input{LatexPreamble}
%%%%%%%%FANCY HEADER%%%%%%%%%
\pagestyle{plain}
\setlength{\headheight}{13.6pt}
\fancyhfoffset[L]{.5in}
%\lhead{\Large \bf{Name:}}
\chead{Executive summary: estimating area}
\rhead{Math 220-2}
%\lfoot{TURN OVER!}
%\rfoot{TURN OVER!}

%%%%%%%PAGE LAYOUT%%%%%%%%%%%%%
\setlength{\textwidth}{6.5in}
\setlength{\textheight}{9in}

%\setlength{\topmargin}{-.8in}
%\setlength{\columnsep}{1.5in}
\addtolength{\hoffset}{-1 in}
\addtolength{\voffset}{-.5 in}


%%%%%%%THEOREM ENVIRONMENTS%%%%%%%%
\theoremstyle{definition}
\newtheorem*{definition}{Definition}
\newtheorem*{definitions}{Definitions}
\newtheorem*{notation}{Notation}
\newtheorem*{example}{Example}
\newtheorem*{comment}{Comment}
\newtheorem*{comments}{Comments}
\newtheorem*{examples}{Examples}
\newtheorem*{warning}{Warning}
\newtheorem*{theorem}{Theorem}
\newtheorem*{corollary}{Corollary}
\newtheorem*{proposition}{Proposition}
\newtheorem*{lemma}{Lemma}

\newtheoremstyle{named}{}{}{}{}{\bfseries}{.}{.5em}{\thmnote{#3}}
\theoremstyle{named}
\newtheorem*{namedtheorem}{Theorem}

\newcounter{myalgctr}
\newenvironment{myalg}{%      define a custom environment
   \bigskip\noindent%         create a vertical offset to previous material
   \refstepcounter{myalgctr}% increment the environment's counter
   \textbf{Algorithm \themyalgctr}% or \textbf, \textit, ...
   \newline%
   }{\par\bigskip}  %
\numberwithin{myalgctr}{section}



\newenvironment{solution}{\begin{proof}[Solution]}{\end{proof}}


%%%%%%%%%%HYPERREFS PACKAGE%%%%%%%%%%%%%%%%%
\usepackage[colorlinks]{hyperref}
%\definecolor{webcolor}{rgb}{0.8,0,0.2}
%\definecolor{webbrown}{rgb}{.6,0,0}
%\usepackage[
%        colorlinks,
%       linkcolor=webbrown,  filecolor=webcolor,  citecolor=webbrown,
%        backref
%]{hyperref}
\usepackage[alphabetic, lite]{amsrefs} % for bibliography
\begin{document}
\thispagestyle{fancy}

\subsection*{Procedures}
\begin{namedtheorem}[Area estimates] Let $f$ be a nonnegative function defined on the interval $[a,b]$, and let $\mathcal{C}$ be the graph of $f$ from $x=a$ to $x=b$. To estimate the area between $\mathcal{C}$ and the $x$-axis proceed as follows:
  \begin{enumerate}[(i),itemsep=0pt,topsep=0pt]
    \item Divide $[a,b]$ into $n$ equal subintervals, each of width $\Delta x=(b-a)/n$.
    \item For each subinterval pick a {\bf sample} input $x^*$ in that interval and build the rectangle whose base is the subinterval and whose height is given by $h=f(x^*)$. The area of this block is
    \[
    A=\underset{\text{height}}{\underbrace{f(x^*)}}\,\underset{\text{width}}{\underbrace{\Delta x}}.
    \]
    \item Sum together the areas of the $n$ blocks constructed in (ii).
  \end{enumerate}
Depending on how the sample inputs $x^*$ are chosen in each subinterval, we get a different estimate. Below you find a number of  common methods.
\begin{itemize}[itemsep=0pt, topsep=0pt]
  \item If $x^*$ is chosen as the left (resp. right) endpoint of each subinterval, the estimate is called a {\bf left sum estimate} (resp. {\bf right sum estimate}).
  \item If $x^*$ is chosen as the midpoint of each subinterval, the estimate is called a {\bf midpoint sum estimate}.
  \item If $x^*$ is chosen so that $f(x^*)$ is the minimum value of $f$ on each subinterval, the estimate is called a {\bf lower sum estimate}.
  \item If $x^*$ is chosen so that $f(x^*)$ is the maximum value of $f$ on each subinterval, the estimate is called an {\bf upper sum estimate}.

\end{itemize}

\end{namedtheorem}
\begin{namedtheorem}[Estimating net change from rate of change] Suppose physical quantity $Q$ is a function of an input $x$, and that $f(x)$ is the {\em instantaneous rate of change} of $Q$ with respect to $x$. The same method used to estimate area under the graph of a function now provides an estimate of the {\em net change of $Q$} as $x$ varies from $a$ to $b$.

\noindent
In this context we think of $r^*=f(x^*)$ as a {\em fixed rate of change} over the given subinterval, in which case an individual term
\[
f(x^*)\, \Delta x
\]
in our sum is understood as an estimate of the net change of $Q$ over the given subinterval under the simplifying assumption that $Q$ changes with {\em fixed rate of change} $r^*=f(x^*)$ over the interval.

\end{namedtheorem}
% %***********************************************
%  \subsection*{Theory}


%*********************************************************
\subsection*{Examples}
\begin{enumerate}
  \item Below you find the graph of the velocity $v(t)$ (in mph) of a driver heading due east $t$ minutes after setting off. Compute an estimate of the area under the graph of $v(t)$ between $t=0$ and $t=20$, and explain what this estimate means physically speaking. Include units!
  \[
  \includegraphics[width=3in]{../Images/EstDistance}
  \]
  \item Let $f(x)=1-x^3$, and let $\mathcal{C}$ be the graph of $f$.
  Compute the upper and lower area estimates of the region between $\mathcal{C}$ and the $x$-axis from $x=0$ to $x=1$ by dividing the interval $[0,1]$ into 4 equal subintervals.

  Draw block pictures of your estimates on the provided graphs.

  Explain why the lower estimate is equal to the right estimate, and why the the upper estimate is equal to the left estimate.
  \[
  \includegraphics[width=2in]{../Images/AreaEstSmallDomain}\  \includegraphics[width=2in]{../Images/AreaEstSmallDomain}
  \]

\item Let $f(x)=1-\val{x}^3$, and let $\mathcal{C}$ be the graph of $f$.

Compute the upper and lower area estimates of the region between $\mathcal{C}$ and the $x$-axis from $x=0$ to $x=1$ by dividing the interval $[-1/2,1]$ into 4 equal subintervals.

Draw block pictures of your estimates on the provided graphs.
\[
\includegraphics[width=3in]{../Images/AreaEstLargeDomain}\  \includegraphics[width=3in]{../Images/AreaEstLargeDomain}
\]
\end{enumerate}




\end{document}
