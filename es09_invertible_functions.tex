\documentclass[11pt]{article}
%%%%%%%%%%PACKAGES%%%%%%%%%%%%%%%%%%%%%%%%%%%%%%%%%%%
\usepackage{latexsym}
\usepackage{amssymb, amsmath, amsthm, amsfonts}
\usepackage{stmaryrd} %For \mapsfrom
%\usepackage[fleqn]{amsmath}  % fleqn option makes aligned equations flushed left!
%\usepackage[english]{babel}
%\usepackage{pgf}
\usepackage{mathtools}
\usepackage[mathscr]{eucal}
\usepackage{fancyhdr}
\usepackage{multicol,parcolumns}
\usepackage{enumerate}
%\usepackage{enumitem}
\usepackage[shortlabels]{enumitem}
\usepackage{graphicx}
\usepackage{extarrows}
\usepackage{cancel}
%\usepackage{tikz}
%\usepackage[all,cmtip]{xy} %\SelectTips{cm}{10}
\usepackage[all]{xy} \SelectTips{cm}{10}
%\usepackage{listings} %For code blocks

\input{LatexPreamble}
%%%%%%%%FANCY HEADER%%%%%%%%%
\pagestyle{plain}
\setlength{\headheight}{13.6pt}
\fancyhfoffset[L]{.5in}
%\lhead{\Large \bf{Name:}}
\chead{Executive summary: antiderivatives}
\rhead{Math 220-2}
%\lfoot{TURN OVER!}
%\rfoot{TURN OVER!}

%%%%%%%PAGE LAYOUT%%%%%%%%%%%%%
\setlength{\textwidth}{6.5in}
\setlength{\textheight}{9in}

%\setlength{\topmargin}{-.8in}
%\setlength{\columnsep}{1.5in}
\addtolength{\hoffset}{-1 in}
\addtolength{\voffset}{-.5 in}


%%%%%%%THEOREM ENVIRONMENTS%%%%%%%%
\theoremstyle{definition}
\newtheorem*{definition}{Definition}
\newtheorem*{definitions}{Definitions}
\newtheorem*{notation}{Notation}
\newtheorem*{example}{Example}
\newtheorem*{comment}{Comment}
\newtheorem*{comments}{Comments}
\newtheorem*{examples}{Examples}
\newtheorem*{warning}{Warning}
\newtheorem*{theorem}{Theorem}
\newtheorem*{corollary}{Corollary}
\newtheorem*{proposition}{Proposition}
\newtheorem*{lemma}{Lemma}

\newtheoremstyle{named}{}{}{}{}{\bfseries}{.}{.5em}{\thmnote{#3}}
\theoremstyle{named}
\newtheorem*{namedtheorem}{Theorem}

\newcounter{myalgctr}
\newenvironment{myalg}{%      define a custom environment
   \bigskip\noindent%         create a vertical offset to previous material
   \refstepcounter{myalgctr}% increment the environment's counter
   \textbf{Algorithm \themyalgctr}% or \textbf, \textit, ...
   \newline%
   }{\par\bigskip}  %
\numberwithin{myalgctr}{section}



\newenvironment{solution}{\begin{proof}[Solution]}{\end{proof}}


%%%%%%%%%%HYPERREFS PACKAGE%%%%%%%%%%%%%%%%%
\usepackage[colorlinks]{hyperref}
%\definecolor{webcolor}{rgb}{0.8,0,0.2}
%\definecolor{webbrown}{rgb}{.6,0,0}
%\usepackage[
%        colorlinks,
%       linkcolor=webbrown,  filecolor=webcolor,  citecolor=webbrown,
%        backref
%]{hyperref}
\usepackage[alphabetic, lite]{amsrefs} % for bibliography
\begin{document}
\thispagestyle{fancy}
\subsection*{Definitions}
\begin{namedtheorem}[One-to-one] A function $f$ is one-to-one on the set $X$ if $f(x_1)\ne f(x_2)$ for all $x_1, x_2\in X$ with $x_1\ne x_2$. We express this with logical notation as
  \[
  x_1\ne x_2\implies f(x_1)\ne f(x_2),
\]
or equivalently, using the contrapositive,
\[
f(x_1)=f(x_2)\implies x_1=x_2.
\]
\end{namedtheorem}



\begin{namedtheorem}[Monotonic functions] Let $f$ be a a real-valued function defined on the set $X$.
  \begin{itemize}
    \item The function $f$ is {\bf increasing on $X$} if $f(x_1)<f(x_2)$ for all $x_1, x_2\in X$ with $x_1<x_2$. Using logical notation:
    \[
    x_1<x_2\implies f(x_1)<f(x_2).
    \]
    \item The function $f$ is {\bf decreasing on $X$} if $f(x_1)>f(x_2)$ for all $x_1, x_2\in X$ with $x_1<x_2$. Using logical notation:
    \[
    x_1<x_2\implies f(x_1)>f(x_2).
    \]
    \item The function $f$ is {\bf monotonic on $X$} if $f$ is increasing on $X$ or $f$ is decreasing on $X$.
  \end{itemize}

\end{namedtheorem}

\begin{namedtheorem}[Inverse function] Suppose $f$ is one-to-one on the set $X$, and let $Y$ be the range of $f$. The {\bf inverse function of $f$} is the function $f^{-1}$ with domain $Y$ defined by the following rule:
  \begin{itemize}
    \item Given $b\in Y$ there is a unique element $a\in X$ such that $f(a)=b$.
    \item We define $f^{-1}(b)=a$.
  \end{itemize}

\end{namedtheorem}

%***********************************************
 \subsection*{Theory}
\begin{namedtheorem}[Horizontal line test] Let $f$ be a real-valued function defined on $X$, and let $\mathcal{C}$ be the graph of $f$ over $X$. The function $f$ is one-to-one on $X$ if and only if for all $c\in\R$ the horizontal line $y=c$ intersects $\mathcal{C}$ in {\em at most} one point.

\end{namedtheorem}
\begin{namedtheorem}[Monotonic functions are one-to-one] If $f$ is monotonic on $X$ then $f$ is invertible on $X$.
\end{namedtheorem}
\begin{namedtheorem}[Inverse function compendium] Let $f$ be one-to-one on its domain $X$, and let $Y$ be the range of $f$. Let $f^{-1}$ be the inverse of $f$.
  \begin{enumerate}[itemsep=0pt]
    \item $f(a)=b$ if and only if $f^{-1}(b)=a$.
    \item The domain of $f^{-1}$ is $Y$, the range of $f$; the range of $f^{-1}$ is $X$, the domain of $f$.
    \item We have
    \begin{align*}
      f^{-1}(f(a))&=a \text{ for all } a\in X\\
      f(f^{-1}(b))&=b \text{ for all } b\in Y.
    \end{align*}
    \item The point $P=(x,y)$ is on the graph of $f$ if and only if the point $Q=(y,x)$ is on the graph of $f^{-1}$.
    \item The graph of $f^{-1}$ is the reflection of the graph of $f$ through the line $y=x$.
  \end{enumerate}

\end{namedtheorem}
\begin{namedtheorem}[Derivative formula for inverses] Assume $f$ is one-to-one and differentiable on the interval $I$, and that $f'(x)\ne 0$ for all $x\in I$. Let $J$ be the range of $f$. Then:
  \begin{enumerate}
    \item The inverse function $f^{-1}$ is differentiable on $J$.
    \item We have
    \[
    (f^{-1})'(b)=\frac{1}{f'(f^{-1}(b))}
    \]
    for all $b\in J$. Alternatively, letting $a$ be the unique element of $D$ such that $f(a)=b$, we have
    \[
    (f^{-1})'(b)=\frac{1}{f'(a)}.
    \]
  \end{enumerate}

\end{namedtheorem}

%*********************************************************
\subsection*{Examples}
\begin{enumerate}
  \item Let $f(x)=x^2+1$.
  \begin{enumerate}
    \item Show that $f$ is not one-to-one on $(-\infty, \infty)$.
    \item Show that $f$ is one-to-one on $(-\infty, 0]$.
    \item Compute a formula for the inverse of $f$ on the domain $(-\infty, 0]$. 
  \end{enumerate}
  \item Let $f(x)=x^5+x^3+3x-5$.
  \begin{enumerate}
    \item Show that $f$ is one-to-one.
    \item Plot three points on the graph of $f^{-1}$.
    \item Compute $(f^{-1})'(-5)$ and $(f^{-1})'(-8)$.
  \end{enumerate}
\end{enumerate}




\end{document}
