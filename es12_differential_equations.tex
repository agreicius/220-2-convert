\documentclass[11pt]{article}
%%%%%%%%%%PACKAGES%%%%%%%%%%%%%%%%%%%%%%%%%%%%%%%%%%%
\usepackage{latexsym}
\usepackage{amssymb, amsmath, amsthm, amsfonts}
\usepackage{stmaryrd} %For \mapsfrom
%\usepackage[fleqn]{amsmath}  % fleqn option makes aligned equations flushed left!
%\usepackage[english]{babel}
%\usepackage{pgf}
\usepackage{mathtools}
\usepackage[mathscr]{eucal}
\usepackage{fancyhdr}
\usepackage{multicol,parcolumns}
\usepackage{enumerate}
%\usepackage{enumitem}
\usepackage[shortlabels]{enumitem}
\usepackage{graphicx}
\usepackage{extarrows}
\usepackage{cancel}
%\usepackage{tikz}
%\usepackage[all,cmtip]{xy} %\SelectTips{cm}{10}
\usepackage[all]{xy} \SelectTips{cm}{10}
%\usepackage{listings} %For code blocks

\input{LatexPreamble}
%%%%%%%%FANCY HEADER%%%%%%%%%
\pagestyle{plain}
\setlength{\headheight}{13.6pt}
\fancyhfoffset[L]{.5in}
%\lhead{\Large \bf{Name:}}
\chead{Executive summary: separable differential equations}
\rhead{Math 220-2}
%\lfoot{TURN OVER!}
%\rfoot{TURN OVER!}

%%%%%%%PAGE LAYOUT%%%%%%%%%%%%%
\setlength{\textwidth}{6.5in}
\setlength{\textheight}{9in}

%\setlength{\topmargin}{-.8in}
%\setlength{\columnsep}{1.5in}
\addtolength{\hoffset}{-1 in}
\addtolength{\voffset}{-.5 in}


%%%%%%%THEOREM ENVIRONMENTS%%%%%%%%
\theoremstyle{definition}
\newtheorem*{definition}{Definition}
\newtheorem*{definitions}{Definitions}
\newtheorem*{notation}{Notation}
\newtheorem*{example}{Example}
\newtheorem*{comment}{Comment}
\newtheorem*{comments}{Comments}
\newtheorem*{examples}{Examples}
\newtheorem*{warning}{Warning}
\newtheorem*{theorem}{Theorem}
\newtheorem*{corollary}{Corollary}
\newtheorem*{proposition}{Proposition}
\newtheorem*{lemma}{Lemma}

\newtheoremstyle{named}{}{}{}{}{\bfseries}{.}{.5em}{\thmnote{#3}}
\theoremstyle{named}
\newtheorem*{namedtheorem}{Theorem}

\newcounter{myalgctr}
\newenvironment{myalg}{%      define a custom environment
   \bigskip\noindent%         create a vertical offset to previous material
   \refstepcounter{myalgctr}% increment the environment's counter
   \textbf{Algorithm \themyalgctr}% or \textbf, \textit, ...
   \newline%
   }{\par\bigskip}  %
\numberwithin{myalgctr}{section}



\newenvironment{solution}{\begin{proof}[Solution]}{\end{proof}}


%%%%%%%%%%HYPERREFS PACKAGE%%%%%%%%%%%%%%%%%
\usepackage[colorlinks]{hyperref}
%\definecolor{webcolor}{rgb}{0.8,0,0.2}
%\definecolor{webbrown}{rgb}{.6,0,0}
%\usepackage[
%        colorlinks,
%       linkcolor=webbrown,  filecolor=webcolor,  citecolor=webbrown,
%        backref
%]{hyperref}
\usepackage[alphabetic, lite]{amsrefs} % for bibliography
\begin{document}
\thispagestyle{fancy}
\subsection*{Definitions}
\begin{namedtheorem}[First-order differential equation] A {\bf first-order differential equation in the unknown $f(x)$} is an equation that can be written in the form
  \[
  f'(x)=F(x,f(x)) \tag{$*$}
  \]
  where $F(x,f(x))$ denotes an arbitrary expression involving $x$ and $f(x)$. Equivalently, a first-order differential equation is an equation that can be written in the form
  \[
  G(x, f'(x), f(x))=0,
  \]
  where $G(x,f(x), f'(x))$ denotes an arbitrary expression involving $x, f(x)$, and $f'(x)$.
\vspace{.1in}
\\
A {\bf solution} to a differential equation is any function $f(x)$ that satisfies this equation. The {\bf general solution} to a differential equation is a formula, possibly containing undetermined constants, describing all solutions to the differential equation.
\end{namedtheorem}
\begin{namedtheorem}[Separable first-order differential equation]A {\bf separable differential equation in the unknown $f(x)$} is a differential equation that can be written in the form
  \[
  f'(x)=g(x)h(f(x)), \text{ or equivalently,  } \frac{dy}{dx}=g(x)h(y),
  \]
  where $y=f(x)$.

\end{namedtheorem}
\begin{namedtheorem}[Exponential growth and decay] Suppose the function $f(x)$ satisfies the equation
  \[
 f'(x)=k f(x),
  \]
  where $k$ is a fixed constant.
  \vspace{.1in}
  \\
  If $k>0$ then $f(x)$ is said to undergo {\bf exponential growth}.
  \vspace{.1in}
  \\
  If $k<0$ then $f(x)$ is said to undergo {\bf exponential decay}.

\end{namedtheorem}
%***********************************************
 % \subsection*{Theory}



%***************************************

\subsection*{Procedures}
\begin{namedtheorem}[Separation of variables (prime form)] To solve a separable differential equation of the form
  \[
  f'(x)=g(x)h(f(x))
  \]
  proceed as follows:
  \begin{enumerate}
    \item {\em Separation}. Write the equation as
    \[
    \frac{f'(x)}{h(f(x))}=g(x).
    \]
    and take take the indefinite integral of both sides.
    \[
    \int \frac{f'(x)}{h(f(x))}\, dx=\int g(x)\, dx.
    \]
    \item {\em Substitution}. Use the substitution $u=f(x)$ to rewrite this equality as
    \[
    \int \frac{1}{h(u)}\, du=\int g(x)\, dx,
    \]
    and attempt to find an antiderivative $F(u)$ of $1/h(u)$ and an antiderivative $G(x)$ for $g(x)$.
    \item {\em Algebra}. Attempt to solve the resulting general equation
    \[
    F(u)=G(x)+C
    \]
    for $u=f(x)$.


  \end{enumerate}

\end{namedtheorem}
\begin{namedtheorem}[Separation of variables (algebraic form)] To solve a separable differential equation of the form
  \[
  \frac{dy}{dx}=g(x)h(y)
  \]
  proceed as follows:
  \begin{enumerate}
    \item {\em Separation}. Write the equation as
    \[
    \frac{1}{h(y)}\, dy=g(x)\, dx
    \]
    and take take the indefinite integral of both sides.
    \[
    \int \frac{1}{h(y)}\, dy=\int g(x)\, dx.
    \]
    \item {\em Integration}. Attempt to find an antiderivative $F(y)$ of $1/h(y)$ and an antiderivative $G(x)$ for $g(x)$.
    \item {\em Algebra}. Attempt to solve the resulting general equation
    \[
    F(y)=G(x)+C
    \]
    for $y$ in terms of $x$.


  \end{enumerate}

\end{namedtheorem}


%*********************************************************
\subsection*{Examples}
\begin{enumerate}
  \item Suppose a hot object cools in a room kept at constant temperature of $T_0$ (in celcius). Newton's law of cooling states that the rate at which the object cools (with respect to time) is proportional to the {\em difference} between its current temperature and the room temperature $T_0$.
  \begin{enumerate}
    \item Write a differential equation that describes Newton's law of cooling in this setting.
    \item Find the general solution to this differential equation.
    \item Find a the particular solution to the situation where $T_0=15$$^\circ$C, the object's initial temperature is $100$$^\circ$C, and after $5$ minutes the object's temperature is $80$$^\circ$C.
  \end{enumerate}
  \item Solve the following differential equations using separation of variables. If an initial condition is given, provide the corresponding particular solution. Otherwise, give the general solution.
  \begin{enumerate}
    \item $\displaystyle f'(x)=xf(x)+x$
    \item $\displaystyle\frac{dy}{dx}=\frac{x^3}{y^2}$
    \item $\cot x\, f'(x)+f(x)=2$, $f(0)=0$.
  \end{enumerate}
\end{enumerate}




\end{document}
