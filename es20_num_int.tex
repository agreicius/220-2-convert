\documentclass[11pt]{article}
%%%%%%%%%%PACKAGES%%%%%%%%%%%%%%%%%%%%%%%%%%%%%%%%%%%
\usepackage{latexsym}
\usepackage{amssymb, amsmath, amsthm, amsfonts}
\usepackage{stmaryrd} %For \mapsfrom
%\usepackage[fleqn]{amsmath}  % fleqn option makes aligned equations flushed left!
%\usepackage[english]{babel}
%\usepackage{pgf}
\usepackage{mathtools}
\usepackage[mathscr]{eucal}
\usepackage{fancyhdr}
\usepackage{multicol,parcolumns}
\usepackage{enumerate}
%\usepackage{enumitem}
\usepackage[shortlabels]{enumitem}
\usepackage{graphicx}
\usepackage{extarrows}
\usepackage{cancel}
%\usepackage{tikz}
%\usepackage[all,cmtip]{xy} %\SelectTips{cm}{10}
\usepackage[all]{xy} \SelectTips{cm}{10}
%\usepackage{listings} %For code blocks

\input{LatexPreamble}
%%%%%%%%FANCY HEADER%%%%%%%%%
\pagestyle{plain}
\setlength{\headheight}{13.6pt}
\fancyhfoffset[L]{.5in}
%\lhead{\Large \bf{Name:}}
\chead{Executive summary: numerical integration}
\rhead{Math 220-2}
%\lfoot{TURN OVER!}
%\rfoot{TURN OVER!}

%%%%%%%PAGE LAYOUT%%%%%%%%%%%%%
\setlength{\textwidth}{6.5in}
\setlength{\textheight}{9in}

%\setlength{\topmargin}{-.8in}
%\setlength{\columnsep}{1.5in}
\addtolength{\hoffset}{-1 in}
\addtolength{\voffset}{-.5 in}


%%%%%%%THEOREM ENVIRONMENTS%%%%%%%%
\theoremstyle{definition}
\newtheorem*{definition}{Definition}
\newtheorem*{definitions}{Definitions}
\newtheorem*{notation}{Notation}
\newtheorem*{example}{Example}
\newtheorem*{comment}{Comment}
\newtheorem*{comments}{Comments}
\newtheorem*{examples}{Examples}
\newtheorem*{warning}{Warning}
\newtheorem*{theorem}{Theorem}
\newtheorem*{corollary}{Corollary}
\newtheorem*{proposition}{Proposition}
\newtheorem*{lemma}{Lemma}

\newtheoremstyle{named}{}{}{}{}{\bfseries}{.}{.5em}{\thmnote{#3}}
\theoremstyle{named}
\newtheorem*{namedtheorem}{Theorem}

\newcounter{myalgctr}
\newenvironment{myalg}{%      define a custom environment
   \bigskip\noindent%         create a vertical offset to previous material
   \refstepcounter{myalgctr}% increment the environment's counter
   \textbf{Algorithm \themyalgctr}% or \textbf, \textit, ...
   \newline%
   }{\par\bigskip}  %
\numberwithin{myalgctr}{section}



\newenvironment{solution}{\begin{proof}[Solution]}{\end{proof}}


%%%%%%%%%%HYPERREFS PACKAGE%%%%%%%%%%%%%%%%%
\usepackage[colorlinks]{hyperref}
%\definecolor{webcolor}{rgb}{0.8,0,0.2}
%\definecolor{webbrown}{rgb}{.6,0,0}
%\usepackage[
%        colorlinks,
%       linkcolor=webbrown,  filecolor=webcolor,  citecolor=webbrown,
%        backref
%]{hyperref}
\usepackage[alphabetic, lite]{amsrefs} % for bibliography
\begin{document}
\thispagestyle{fancy}


%***********************************************




%***************************************

\subsection*{Definitions}
\begin{namedtheorem}[Trapezoidal rule] Let $f$ be an integrable function on $[a,b]$, let $n$ be a positive integer, and let
  \[
  a=x_0< x_1< x_2<\cdots < x_n=b
  \]
  be partition of $[a,b]$ into $n$ subintervals of equal length $\Delta x=\frac{b-a}{n}$.
  \vspace{.1in}
  \\
  The {\bf $n$-th trapezoidal estimate} of $\displaystyle\int f(x)\, dx$, denoted $T_n$, is defined as
  \[
  T_n=\frac{1}{2}\Delta x(f(x_0)+2f(x_1)+2f(x_2)+\cdots +2f(x_{n-1})+f(x_n))\approx \int_a^b f(x)\, dx.
  \]
  The trapezoidal estimate is the result of approximating the graph of $f$ with the polygon passing through the points $P_0=(x_0, f(x_0)), P_1=(x_1,f(x_1)), \dots, P_n=(x_n, f(x_n))$.
\end{namedtheorem}
\begin{namedtheorem}[Simpson's rule]  Let $f$ be an integrable function on $[a,b]$, let $n=2r$ be an even positive integer, and let
  \[
  a=x_0< x_1< x_2<\cdots < x_n=b
  \]
  be partition of $[a,b]$ into $n$ subintervals of equal length $\Delta x=\frac{b-a}{n}$.
  \vspace{.1in}
  \\
  The {\bf $n$-th Simpson's rule estimate} of $\displaystyle\int f(x)\, dx$, denoted $S_n$, is defined as
  \[
  S_n=\frac{1}{3}\Delta x(f(x_0)+4f(x_1)+2f(x_2)+\cdots +2f(x_{n-2})+4f(x_{n-1})+f(x_n))\approx \int_a^b f(x)\, dx.
  \]
  The Simpson's rule estimate is the result of approximating the graph of $f$ over each of the $r$ subintervals $[x_{2(k-1)},x_{2k}]$ with the unique ``parabolic arc"\footnote{If the three points happen to be colinear, then the ``parabolic arc" will actually be a line.} passing through $P_{2(k-1)}=(x_{2(k-1)}, f(x_{2(k-1)}))$, $P_{2k-1}=(x_{2k-1}, f(x_{2k-1})), P_{2k}=(x_{2k},f(x_{2k}))$.
\end{namedtheorem}

%************************
\subsection*{Theory}
\begin{namedtheorem}[Error estimates] Let $f$ be an integrable function on $[a,b]$, let $n$ be a positive integer, and let
  \[
  a=x_0< x_1< x_2<\cdots < x_n=b.
  \]
  be partition of $[a,b]$ into $n$ subintervals of equal length $\Delta x=\frac{b-a}{n}$.
  \begin{enumerate}
    \item Let $RS_n$ be either the right or left Riemann sum for this partition. Suppose $\vert f'(x)\vert \leq M$ for all $x$ in $[a,b]$. Then
    \[
    \left\vert\int_a^b f(x)\, dx - RS_n\right\vert\leq \frac{M(b-a)^2}{2n}.
    \]
    \item Let $T_n$ be the $n$-th trapezoidal estimate of $\int_a^b f(x)\, dx$. Suppose $\vert f''(x)\vert\leq N$ for all $x$ in $[a,b]$. Then
    \[
    \left\vert\int_a^b f(x)\, dx - T_n\right\vert\leq\frac{N(b-a)^3}{12n^2}.
    \]
    \item Suppose $n$ is even, and let $S_n$ be the $n$-th Simpson's rule estimate of $\int_a^bf(x)\, dx$. Suppose $\vert f^{(4)}(x)\vert\leq K$ for all $x$ in $[a,b]$. Then
    \[
    \left\vert\int_a^b f(x)\, dx - S_n\right\vert\leq\frac{K(b-a)^5}{180n^4}.
    \]
  \end{enumerate}

\end{namedtheorem}

%*********************************************************
\subsection*{Examples}
\begin{enumerate}
  \item Let $f(x)=\frac{1}{x}$. Recall that we have by definition $\ln 4=\int_1^4f(x)\, dx$. Compute (a) the $n=6$ trapezoidal estimate of $I$, and (b) the $n=6$ Simpson's rule estimate of $I$.
  \item Let $f(x)=\frac{4}{x^2+1}$, and let $I=\int_0^1f(x)\, dx$. Observe that $I=4(\arctan(1)-\arctan 0)=\pi$.

  Compute (a) the $n=6$ trapezoidal estimate of $I$, and (b) the $n=6$ Simpson's rule estimate of $I$.

  \item Compute bounds for the errors in (a) the $n=10$ trapezoidal estimate of $\ln 4$ and (b) the $n=10$ Simpson's rule estimate of $\ln 4$.

  \item Compute bounds for the errors in (a) the $n=10$ trapezoidal estimate of $\pi=\int_0^14/(x^2+1)\, dx$ and (b) the $n=10$ Simpson's rule estimate of $\pi=\int_0^14/(x^2+1)\, dx$.

  {\bf Hint}. Letting $f(x)=4/(x^2+1)$, we have
  \begin{align*}
    f''(x)&=\frac{8(3x^2-1)}{(x^2+1)^3}\\
    f^{(4)}(x)&=\frac{96(5x^4-10x^2+1)}{(x^2+1)^5}.
  \end{align*}
  \item Find (a) an $n$ such that the $n$-th trapezoidal estimate of $\pi=\int_0^14/(x^2+1)\, dx$ is within $10^{-9}$ of the actual value, and (b) an $n$ such that the $n$-th Simpson's rule estimate of $\pi=\int_0^14/(x^2+1)\, dx$ is within $10^{-9}$ of the actual value.
\end{enumerate}




\end{document}
