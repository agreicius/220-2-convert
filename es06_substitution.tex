\documentclass[11pt]{article}
%%%%%%%%%%PACKAGES%%%%%%%%%%%%%%%%%%%%%%%%%%%%%%%%%%%
\usepackage{latexsym}
\usepackage{amssymb, amsmath, amsthm, amsfonts}
\usepackage{stmaryrd} %For \mapsfrom
%\usepackage[fleqn]{amsmath}  % fleqn option makes aligned equations flushed left!
%\usepackage[english]{babel}
%\usepackage{pgf}
\usepackage{mathtools}
\usepackage[mathscr]{eucal}
\usepackage{fancyhdr}
\usepackage{multicol,parcolumns}
\usepackage{enumerate}
%\usepackage{enumitem}
\usepackage[shortlabels]{enumitem}
\usepackage{graphicx}
\usepackage{extarrows}
\usepackage{cancel}
%\usepackage{tikz}
%\usepackage[all,cmtip]{xy} %\SelectTips{cm}{10}
\usepackage[all]{xy} \SelectTips{cm}{10}
%\usepackage{listings} %For code blocks

\input{LatexPreamble}
%%%%%%%%FANCY HEADER%%%%%%%%%
\pagestyle{plain}
\setlength{\headheight}{13.6pt}
\fancyhfoffset[L]{.5in}
%\lhead{\Large \bf{Name:}}
\chead{Executive summary: substitution}
\rhead{Math 220-2}
%\lfoot{TURN OVER!}
%\rfoot{TURN OVER!}

%%%%%%%PAGE LAYOUT%%%%%%%%%%%%%
\setlength{\textwidth}{6.5in}
\setlength{\textheight}{9in}

%\setlength{\topmargin}{-.8in}
%\setlength{\columnsep}{1.5in}
\addtolength{\hoffset}{-1 in}
\addtolength{\voffset}{-.5 in}


%%%%%%%THEOREM ENVIRONMENTS%%%%%%%%
\theoremstyle{definition}
\newtheorem*{definition}{Definition}
\newtheorem*{definitions}{Definitions}
\newtheorem*{notation}{Notation}
\newtheorem*{example}{Example}
\newtheorem*{comment}{Comment}
\newtheorem*{comments}{Comments}
\newtheorem*{examples}{Examples}
\newtheorem*{warning}{Warning}
\newtheorem*{theorem}{Theorem}
\newtheorem*{corollary}{Corollary}
\newtheorem*{proposition}{Proposition}
\newtheorem*{lemma}{Lemma}

\newtheoremstyle{named}{}{}{}{}{\bfseries}{.}{.5em}{\thmnote{#3}}
\theoremstyle{named}
\newtheorem*{namedtheorem}{Theorem}

\newcounter{myalgctr}
\newenvironment{myalg}{%      define a custom environment
   \bigskip\noindent%         create a vertical offset to previous material
   \refstepcounter{myalgctr}% increment the environment's counter
   \textbf{Algorithm \themyalgctr}% or \textbf, \textit, ...
   \newline%
   }{\par\bigskip}  %
\numberwithin{myalgctr}{section}



\newenvironment{solution}{\begin{proof}[Solution]}{\end{proof}}


%%%%%%%%%%HYPERREFS PACKAGE%%%%%%%%%%%%%%%%%
\usepackage[colorlinks]{hyperref}
%\definecolor{webcolor}{rgb}{0.8,0,0.2}
%\definecolor{webbrown}{rgb}{.6,0,0}
%\usepackage[
%        colorlinks,
%       linkcolor=webbrown,  filecolor=webcolor,  citecolor=webbrown,
%        backref
%]{hyperref}
\usepackage[alphabetic, lite]{amsrefs} % for bibliography
\begin{document}
\thispagestyle{fancy}
\subsection*{Definitions}



%***********************************************
 \subsection*{Theory}

\begin{namedtheorem}[Chain rule (antiderivative version)] Let $u$ be a differentiable function on its domain, and suppose $f$ is continuous on the range of $u$. Suppose $F(x)$ is an antiderivative of $f(x)$. Then $F(u(x))$ is an antiderivative of $f(u(x))u'(x)$: i.e.,
  \[
  \int f(u(x))u'(x)\, dx=F(u(x))+C.
  \]
  Alternatively, letting $u=u(x)$ we have
  \[
  \int f(u(x))u'(x)\, dx=\int f(u)\, du.
  \]
\end{namedtheorem}
\begin{comment}
Before seeing how to correctly use the chain rule (antiderivative version) to compute indefinite integrals, it is worthwhile noting a tempting, but {\em incorrect} method: namely, if $F(x)$ is an antiderivative of $f(x)$ it is not in general true that $F(u(x))$ is an antiderivative of $f(u(x))$. Indeed, the chain rule tells us that $F(u(x))$ is an antiderivative of $f(u(x))u'(x)$.

\end{comment}


%***************************************
\subsection*{Procedures}
\begin{namedtheorem}[Substitution technique (indefinite integrals)] We wish to compute
$
\displaystyle\int f(x)\, dx.
$
\begin{enumerate}[itemsep=0pt, topsep=0pt]
  \item Pick a differentiable substitution function $u=u(x)$. Set
  \begin{align}
    u&=u(x)\\
    du&=u'(x)\, dx
  \end{align}
  \item Algebraically manipulate equations (1) and (2) to find a function $g$ such that
  \[
  f(x)\, dx=g(u)\, du.
  \]
  By the chain rule (antiderivative form) we have
  \[
  \int f(x)\, dx=\int g(u)\, du.
  \]
  \item If possible, find an antiderivative $G$ of $g$. Then $F(x)=G(u(x))$ is an antiderivative of $f(x)$: i.e.,
  \[
  \int f(x)\, dx=G(u(x))+C
  \]
\end{enumerate}

\end{namedtheorem}
\begin{comment}
There is no such thing as a {\em correct} or {\em incorrect} substitution, and you are encouraged to be creative with your choice of substitution $u(x)$. Instead think of a substitution as either {\em helpful} or {\em not helpful} (or possibly {\em somewhat helpful}). The success of a particular choice of $u(x)$ depends on two factors:
\begin{enumerate}[itemsep=0pt, topsep=0pt]
  \item Can you algebraically find a function $g$ such that $f(x)=g(u(x))u'(x)$?
  \item Having found a suitable $g$, can you find an antiderivative $G$ of $g$?
\end{enumerate}
\end{comment}
\newpage
\begin{namedtheorem}[Substitution technique (definite integrals)]
We wish to compute the definite integral $\displaystyle\int_a^b f(x)\, dx$ using a substitution $u=u(x)$. We can proceed in two different ways.
\begin{enumerate}[itemsep=0pt]
  \item {\bf Two-step method}. {\em First} find an antiderivative $F(x)$ of $f(x)$ using the substitution method for indefinite integrals, then use the FTC to compute $\displaystyle\int_a^bf(x)\, dx=F(b)-F(a)$.
  \item {\bf Streamlined method}. Find the $g$ such that $f(x)\, dx=g(u)\, du$ (as with indefinite integral substitution) then convert the original definite integral into a new definite integral with respect to $u$ by also {\em changing the limits of integration}:
  \[
  \int_{x=a}^{x=b}f(x)\, dx=\int_{u=u(a)}^{u=u(b)}g(u)\, du.
  \]
\end{enumerate}
\end{namedtheorem}

%*********************************************************
\subsection*{Examples}
\begin{enumerate}
  \item {\bf More or less obvious substitutions}. Use the substitution technique to compute the following indefinite integrals.
  \begin{enumerate}
    \item $\displaystyle\int x^2\sqrt{x^3+1}\, dx$
    \item $\displaystyle\int -\sin t\, \sqrt{\cos t}\, dt$
    \item $\displaystyle\int \frac{\sin(\sqrt{u})}{\sqrt{u}}\, du$

  \end{enumerate}
  \item {\bf Less obvious substitutions}. Use the substitution technique to compute the following indefinite integrals.
  \begin{enumerate}
    \item $\displaystyle\int\frac{x}{\sqrt{x+1}}\, dx$
    \item $\displaystyle\int (1+\sqrt{t})^{100}\, dt$
  \end{enumerate}
\item {\bf Substitution with definite integrals}. Use the substitution technique to compute the following definite integrals. You may use either the two-step or streamlined method.
\begin{enumerate}
  \item $\displaystyle\int_{\pi}^{2\pi}\cos^2(x)\sin x\, dx$
  \item $\displaystyle\int_{1}^{2} \sqrt{s^8+s^6}\, ds$
\end{enumerate}
\end{enumerate}



\end{document}
