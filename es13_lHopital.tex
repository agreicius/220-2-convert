\documentclass[11pt]{article}
%%%%%%%%%%PACKAGES%%%%%%%%%%%%%%%%%%%%%%%%%%%%%%%%%%%
\usepackage{latexsym}
\usepackage{amssymb, amsmath, amsthm, amsfonts}
\usepackage{stmaryrd} %For \mapsfrom
%\usepackage[fleqn]{amsmath}  % fleqn option makes aligned equations flushed left!
%\usepackage[english]{babel}
%\usepackage{pgf}
\usepackage{mathtools}
\usepackage[mathscr]{eucal}
\usepackage{fancyhdr}
\usepackage{multicol,parcolumns}
\usepackage{enumerate}
%\usepackage{enumitem}
\usepackage[shortlabels]{enumitem}
\usepackage{graphicx}
\usepackage{extarrows}
\usepackage{cancel}
%\usepackage{tikz}
%\usepackage[all,cmtip]{xy} %\SelectTips{cm}{10}
\usepackage[all]{xy} \SelectTips{cm}{10}
%\usepackage{listings} %For code blocks

\input{LatexPreamble}
%%%%%%%%FANCY HEADER%%%%%%%%%
\pagestyle{plain}
\setlength{\headheight}{13.6pt}
\fancyhfoffset[L]{.5in}
%\lhead{\Large \bf{Name:}}
\chead{Executive summary: l'H\^opital's rule}
\rhead{Math 220-2}
%\lfoot{TURN OVER!}
%\rfoot{TURN OVER!}

%%%%%%%PAGE LAYOUT%%%%%%%%%%%%%
\setlength{\textwidth}{6.5in}
\setlength{\textheight}{9in}

%\setlength{\topmargin}{-.8in}
%\setlength{\columnsep}{1.5in}
\addtolength{\hoffset}{-1 in}
\addtolength{\voffset}{-.5 in}


%%%%%%%THEOREM ENVIRONMENTS%%%%%%%%
\theoremstyle{definition}
\newtheorem*{definition}{Definition}
\newtheorem*{definitions}{Definitions}
\newtheorem*{notation}{Notation}
\newtheorem*{example}{Example}
\newtheorem*{comment}{Comment}
\newtheorem*{comments}{Comments}
\newtheorem*{examples}{Examples}
\newtheorem*{warning}{Warning}
\newtheorem*{theorem}{Theorem}
\newtheorem*{corollary}{Corollary}
\newtheorem*{proposition}{Proposition}
\newtheorem*{lemma}{Lemma}

\newtheoremstyle{named}{}{}{}{}{\bfseries}{.}{.5em}{\thmnote{#3}}
\theoremstyle{named}
\newtheorem*{namedtheorem}{Theorem}

\newcounter{myalgctr}
\newenvironment{myalg}{%      define a custom environment
\bigskip\noindent%         create a vertical offset to previous material
\refstepcounter{myalgctr}% increment the environment's counter
\textbf{Algorithm \themyalgctr}% or \textbf, \textit, ...
\newline%
}{\par\bigskip}  %
\numberwithin{myalgctr}{section}



\newenvironment{solution}{\begin{proof}[Solution]}{\end{proof}}


%%%%%%%%%%HYPERREFS PACKAGE%%%%%%%%%%%%%%%%%
\usepackage[colorlinks]{hyperref}
%\definecolor{webcolor}{rgb}{0.8,0,0.2}
%\definecolor{webbrown}{rgb}{.6,0,0}
%\usepackage[
%        colorlinks,
%       linkcolor=webbrown,  filecolor=webcolor,  citecolor=webbrown,
%        backref
%]{hyperref}
\usepackage[alphabetic, lite]{amsrefs} % for bibliography
\begin{document}
\thispagestyle{fancy}
\subsection*{Definitions}
\begin{namedtheorem}[Indeterminate forms]Consider a limit expression of the form
  \[
  \lim_{x\to a}\frac{f(x)}{g(x)},
  \]
  where $a$ is either a finite number or $\pm\infty$.
  \vspace{.1in}
  \\
  The expression is an {\bf indeterminate form of type $0/0$} if
  \[
  \lim_{x\to a}f(x)=\lim_{x\to a}g(x)=0.
  \]
  The expression is an {\bf indeterminate form of type $\infty/\infty$} if
  \[
  \lim_{x\to a}f(x)=\pm\infty \text{ and } \lim_{x\to a}g(x)=\pm\infty.
  \]

\end{namedtheorem}
\begin{comment}
  A limit expression having an indeterminate form does {\em not} mean that the limit does not exist. You should interpret this conclusion as saying simply that our current analysis is not detailed enough to determine whether the limit exists and/or what that limit is.
  \vspace{.1in}
  \\
  In this spirit, we will be careful {\em not} to write
  \[
  \lim_{x\to a}\frac{f(x)}{g(x)}=\frac{0}{0} \text{ or } \lim_{x\to a}\frac{f(x)}{g(x)}=\frac{\infty}{\infty},
  \]
  as this suggests we are asserting something more definitive about the limit.

\end{comment}
\begin{namedtheorem}[Further indeterminate forms] Assume $a$ is either a finite number or $\pm\infty$.
  \vspace{.1in}
  \\
  If $\displaystyle\lim_{x\to a}f(x)=\lim_{x\to a}g(x)=\infty$, then
  $\displaystyle\lim_{x\to a}f(x)-g(x)$ is an {\bf indeterminate form of type $\infty-\infty$}.
  \vspace{.1in}
  \\
  If $\displaystyle\lim_{x\to a}f(x)=0$ and $\lim_{x\to a}g(x)=\pm\infty$, then
  $\displaystyle\lim_{x\to a}f(x)g(x)$ is an {\bf indeterminate \\ form of type $0\cdot\infty$}.
  \vspace{.1in}
  \\
  If $\displaystyle\lim_{x\to a}f(x)=\lim_{x\to a}g(x)=0$, then
  $\displaystyle\lim_{x\to a}f(x)^{g(x)}$ is an {\bf indeterminate form of type $0^0$}.
  \vspace{.1in}
  \\
  If $\displaystyle\lim_{x\to a}f(x)=\infty$ and $\lim_{x\to a}g(x)=0$, then
  $\displaystyle\lim_{x\to a}f(x)^{g(x)}$ is an {\bf indeterminate form  of type $\infty^0$}.
  \vspace{.1in}
  \\
  If $\displaystyle\lim_{x\to a}f(x)=1$ and $\lim_{x\to a}g(x)=\infty$, then
  $\displaystyle\lim_{x\to a}f(x)^{g(x)}$ is an {\bf indeterminate form of type $1^\infty$}.

\end{namedtheorem}

%***********************************************
\subsection*{Theory}
\begin{namedtheorem}[L'H\^opital's rule] Let $f$ and $g$ be differentiable on an open interval $I$ containing $a$, where $a$ is either a finite number or $\pm\infty$, and suppose $g'(x)\ne 0$ for all $x\ne a$ in the interval.
  \vspace{.1in}
  \\
  If $\displaystyle\lim_{x\to a}\frac{f(x)}{g(x)}$ is an indeterminate form of type $0/0$ or $\infty/\infty$, then
  \[
  \lim_{x\to a}\frac{f(x)}{g(x)}=\lim_{x\to a}\frac{f'(x)}{g'(x)},
  \]
  provided the limit on the right exists or is equal to $\pm\infty$. \\
  The same result holds if we replace the limit with a one-sided limit.

\end{namedtheorem}
\begin{comment}
Students tend to fall madly in love with l'H\^opital's rule after seeing it for the first time. Some comments to temper your passion:
\begin{enumerate}
  \item Make sure the necessary conditions hold: (a) $f,g$ differentiable on an interval about $a$, $g(x)\ne 0$ on for $x\ne a$, and the limit expression is indeterminate of type $0/0$ or $\infty/\infty$.
  \item As magic as the rule appears, there are many examples where either the application of this rule does not help, and/or it is easier to use a different technique. Consider the following limits, for example:
  \[
  \lim_{x\to\infty}\frac{e^x+e^{-x}}{e^x-e^{-x}}, \hspace{10pt} \lim_{x\rightarrow \infty}\frac{x^4-x^2+5x+7}{2x^4+x^3+x^2+x+1}
  \]
\end{enumerate}
\end{comment}

%***************************************

\subsection*{Procedures}

%*********************************************************
\subsection*{Examples}
\begin{enumerate}
  \item Decide whether the following limit expressions have determinate or indeterminate forms. If determinate, compute the limit.
  \begin{enumerate}
    \item $\displaystyle\lim_{x\to 0+}\frac{\sin x}{\ln x}$
    \item $\displaystyle\lim_{x\to (\pi/2)^-}\frac{\tan x}{\cos x}$
    \item $\displaystyle\lim_{x\to\infty}\frac{e^x}{2^x+3^x}$
  \end{enumerate}
  \item Compute the following limits.
  \begin{enumerate}
    \item $\displaystyle\lim_{x\to\infty}\frac{\ln x}{x^{1000}}$
    \item $\displaystyle\lim_{x\to 0}\frac{2^x-3^{-x}}{4^x-5^{-x}}$
    \item $\displaystyle\lim_{x\to 1}\frac{\cos(\pi x/2)}{\log_2(x)}$
    \item $\displaystyle\lim_{x\to 0}\frac{x-\sin x}{x\sin x}$
  \end{enumerate}
  \item Compute the following limits.
  \begin{enumerate}
    \item $\displaystyle\lim_{x\to 0^+}\frac{1}{\sin x}-\frac{1}{x}$
    \item $\displaystyle\lim_{x\to \infty}2x-\sqrt{4x^2-13x}$
    \item $\displaystyle\lim_{x\to -\infty}x^22^{x}$
    \item $\displaystyle\lim_{x\to 0^+}(1+x)^{1/x}$
    \item $\displaystyle\lim_{x\to \infty}(1+x^2)^{2/x}$
  \end{enumerate}
\end{enumerate}




\end{document}
