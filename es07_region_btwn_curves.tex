\documentclass[11pt]{article}
%%%%%%%%%%PACKAGES%%%%%%%%%%%%%%%%%%%%%%%%%%%%%%%%%%%
\usepackage{latexsym}
\usepackage{amssymb, amsmath, amsthm, amsfonts}
\usepackage{stmaryrd} %For \mapsfrom
%\usepackage[fleqn]{amsmath}  % fleqn option makes aligned equations flushed left!
%\usepackage[english]{babel}
%\usepackage{pgf}
\usepackage{mathtools}
\usepackage[mathscr]{eucal}
\usepackage{fancyhdr}
\usepackage{multicol,parcolumns}
\usepackage{enumerate}
%\usepackage{enumitem}
\usepackage[shortlabels]{enumitem}
\usepackage{graphicx}
\usepackage{extarrows}
\usepackage{cancel}
%\usepackage{tikz}
%\usepackage[all,cmtip]{xy} %\SelectTips{cm}{10}
\usepackage[all]{xy} \SelectTips{cm}{10}
%\usepackage{listings} %For code blocks

\input{LatexPreamble}
%%%%%%%%FANCY HEADER%%%%%%%%%
\pagestyle{plain}
\setlength{\headheight}{13.6pt}
\fancyhfoffset[L]{.5in}
%\lhead{\Large \bf{Name:}}
\chead{Executive summary: regions between curves}
\rhead{Math 220-2}
%\lfoot{TURN OVER!}
%\rfoot{TURN OVER!}

%%%%%%%PAGE LAYOUT%%%%%%%%%%%%%
\setlength{\textwidth}{6.5in}
\setlength{\textheight}{9in}

%\setlength{\topmargin}{-.8in}
%\setlength{\columnsep}{1.5in}
\addtolength{\hoffset}{-1 in}
\addtolength{\voffset}{-.5 in}


%%%%%%%THEOREM ENVIRONMENTS%%%%%%%%
\theoremstyle{definition}
\newtheorem*{definition}{Definition}
\newtheorem*{definitions}{Definitions}
\newtheorem*{notation}{Notation}
\newtheorem*{example}{Example}
\newtheorem*{comment}{Comment}
\newtheorem*{comments}{Comments}
\newtheorem*{examples}{Examples}
\newtheorem*{warning}{Warning}
\newtheorem*{theorem}{Theorem}
\newtheorem*{corollary}{Corollary}
\newtheorem*{proposition}{Proposition}
\newtheorem*{lemma}{Lemma}

\newtheoremstyle{named}{}{}{}{}{\bfseries}{.}{.5em}{\thmnote{#3}}
\theoremstyle{named}
\newtheorem*{namedtheorem}{Theorem}

\newcounter{myalgctr}
\newenvironment{myalg}{%      define a custom environment
   \bigskip\noindent%         create a vertical offset to previous material
   \refstepcounter{myalgctr}% increment the environment's counter
   \textbf{Algorithm \themyalgctr}% or \textbf, \textit, ...
   \newline%
   }{\par\bigskip}  %
\numberwithin{myalgctr}{section}



\newenvironment{solution}{\begin{proof}[Solution]}{\end{proof}}


%%%%%%%%%%HYPERREFS PACKAGE%%%%%%%%%%%%%%%%%
\usepackage[colorlinks]{hyperref}
%\definecolor{webcolor}{rgb}{0.8,0,0.2}
%\definecolor{webbrown}{rgb}{.6,0,0}
%\usepackage[
%        colorlinks,
%       linkcolor=webbrown,  filecolor=webcolor,  citecolor=webbrown,
%        backref
%]{hyperref}
\usepackage[alphabetic, lite]{amsrefs} % for bibliography
\begin{document}
\thispagestyle{fancy}
\subsection*{Definitions}
\begin{namedtheorem}[Area of region between curves] Suppose $f(x)\geq g(x)$ for all $x\in [a,b]$. Let $\mathcal{C}_1$ be the graph of $f$, let $\mathcal{C}_2$ be the graph of $g$, and let $\mathcal{R}$ be the region between $\mathcal{C}_1$ and $\mathcal{C}_2$ lying over the interval $[a,b]$ on the $x$-axis. We define the area of $\mathcal{R}$ to be the integral of $f-g$ over $[a,b]$: i.e.,
  \[
  \text{area}(\mathcal{R})=\int_a^b f(x)-g(x)\, dx.
  \]
Similarly, suppose $x=p(y)$ and $x=p(q)$ are two functions of $y$ satisfying $p(y)\geq q(y)$ for all $y\in [c,d]$. Let $\mathcal{C}_1$ be the graph of $p$, let $\mathcal{C}_2$ be the graph of $q$, and let $\mathcal{R}$ be the region between $\mathcal{C}_1$ and $\mathcal{C}_2$ lying over the interval $[c,d]$ on the $y$-axis. We define the area of $\mathcal{R}$ to be the integral of $p-q$ over $[c,d]$: i.e.,
  \[
  \text{area}(\mathcal{R})=\int_c^d p(y)-q(y)\, dy.
  \]
\end{namedtheorem}
\begin{comment}
Observe that the definition only applies when $f(x)\geq g(x)$ for all $x$ in the given interval. This ensures that the area of $\mathcal{R}$, as defined, is at least nonnegative.
\end{comment}
%***********************************************
 \subsection*{Theory}

\begin{namedtheorem}[Graphical argument in support of area definition] Suppose $f(x)\geq g(x)$ for all $x\in [a,b]$. Let $\mathcal{C}_1$ be the graph of $f$, let $\mathcal{C}_2$ be the graph of $g$, and $\mathcal{R}$ be the region between $\mathcal{C}_1$ and $\mathcal{C}_2$ over the interval $[a,b]$ on the $x$-axis.
  \begin{enumerate}
    \item Suppose we also have $f(x)\geq g(x)\geq 0$ for all $x\in [a,b]$. Then we have
    \begin{align*}
      \text{area}(\mathcal{R})&=\int_a^b f(x)-g(x)\, dx \\
      &=\int_a^b f(x)\, dx -\int_a^bg(x)\, dx\\
      &=\text{area}(\mathcal{R}_1)-\text{area}(\mathcal{R}_2),
    \end{align*}
    where $\mathcal{R}_i$ is the region lying between $\mathcal{C}_i$ and the $x$-axis over the interval $[a,b]$. Intuitively, this difference of areas should indeed be the area between the two curves.
    \item To reduce the general case $f(x)\geq g(x)$ to the case above, simply shift both functions (and hence also $\mathcal{R}$) up by a large enough constant $C$ so that $f(x)\geq g(x)\geq 0$.  This operation does not affect the area of $\mathcal{R}$, and the $C$ gets canceled in the integral computation thanks to the difference operator!
  \end{enumerate}

\end{namedtheorem}

%***************************************
\subsection*{Procedures}
\begin{namedtheorem}[Regions between intertwined curves] Suppose $f$ and $g$ are continuous on the interval $[a,b]$ and intersect one another finitely many times. Let $\mathcal{R}$ be the region between the graphs of $f$ and $g$ lying over the interval $[a,b]$. To compute the area of $\mathcal{R}$, proceed as follows:
  \begin{enumerate}
    \item Parition $[a,b]$ into subintervals for which one of the functions is always greater than or equal to the other.
    \item On each such subinterval compute the area of the corresponding region by integrating the appropriate difference.
    \item Sum up the areas you compute in (2).
  \end{enumerate}


\end{namedtheorem}


%*********************************************************
\subsection*{Examples}
\begin{enumerate}
  \item Let $\mathcal{R}$ be the region between the parabola $x+y^2=4$ and the line $2x+y=2$ lying in the first quadrant. Compute the are of $\mathcal{R}$.

  You may do this either by thinking of the curves as graphs of functions of $x$, or graphs of functions of $y$. Which approach is easier?
  \item Compute the area of the region between the parabolas $y=-x^2-2x$ and $y=x^2-4$ lying within the lines $x=-3$ and $x=2$. 
\end{enumerate}



\end{document}
