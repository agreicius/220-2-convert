\documentclass[11pt]{article}
%%%%%%%%%%PACKAGES%%%%%%%%%%%%%%%%%%%%%%%%%%%%%%%%%%%
\usepackage{latexsym}
\usepackage{amssymb, amsmath, amsthm, amsfonts}
\usepackage{stmaryrd} %For \mapsfrom
%\usepackage[fleqn]{amsmath}  % fleqn option makes aligned equations flushed left!
%\usepackage[english]{babel}
%\usepackage{pgf}
\usepackage{mathtools}
\usepackage[mathscr]{eucal}
\usepackage{fancyhdr}
\usepackage{multicol,parcolumns}
\usepackage{enumerate}
%\usepackage{enumitem}
\usepackage[shortlabels]{enumitem}
\usepackage{graphicx}
\usepackage{extarrows}
\usepackage{cancel}
%\usepackage{tikz}
%\usepackage[all,cmtip]{xy} %\SelectTips{cm}{10}
\usepackage[all]{xy} \SelectTips{cm}{10}
%\usepackage{listings} %For code blocks

\input{LatexPreamble}
%%%%%%%%FANCY HEADER%%%%%%%%%
\pagestyle{plain}
\setlength{\headheight}{13.6pt}
\fancyhfoffset[L]{.5in}
%\lhead{\Large \bf{Name:}}
\chead{Executive summary: partial fraction decomposition}
\rhead{Math 220-2}
%\lfoot{TURN OVER!}
%\rfoot{TURN OVER!}

%%%%%%%PAGE LAYOUT%%%%%%%%%%%%%
\setlength{\textwidth}{6.5in}
\setlength{\textheight}{9in}

%\setlength{\topmargin}{-.8in}
%\setlength{\columnsep}{1.5in}
\addtolength{\hoffset}{-1 in}
\addtolength{\voffset}{-.5 in}


%%%%%%%THEOREM ENVIRONMENTS%%%%%%%%
\theoremstyle{definition}
\newtheorem*{definition}{Definition}
\newtheorem*{definitions}{Definitions}
\newtheorem*{notation}{Notation}
\newtheorem*{example}{Example}
\newtheorem*{comment}{Comment}
\newtheorem*{comments}{Comments}
\newtheorem*{examples}{Examples}
\newtheorem*{warning}{Warning}
\newtheorem*{theorem}{Theorem}
\newtheorem*{corollary}{Corollary}
\newtheorem*{proposition}{Proposition}
\newtheorem*{lemma}{Lemma}

\newtheoremstyle{named}{}{}{}{}{\bfseries}{.}{.5em}{\thmnote{#3}}
\theoremstyle{named}
\newtheorem*{namedtheorem}{Theorem}

\newcounter{myalgctr}
\newenvironment{myalg}{%      define a custom environment
   \bigskip\noindent%         create a vertical offset to previous material
   \refstepcounter{myalgctr}% increment the environment's counter
   \textbf{Algorithm \themyalgctr}% or \textbf, \textit, ...
   \newline%
   }{\par\bigskip}  %
\numberwithin{myalgctr}{section}



\newenvironment{solution}{\begin{proof}[Solution]}{\end{proof}}


%%%%%%%%%%HYPERREFS PACKAGE%%%%%%%%%%%%%%%%%
\usepackage[colorlinks]{hyperref}
%\definecolor{webcolor}{rgb}{0.8,0,0.2}
%\definecolor{webbrown}{rgb}{.6,0,0}
%\usepackage[
%        colorlinks,
%       linkcolor=webbrown,  filecolor=webcolor,  citecolor=webbrown,
%        backref
%]{hyperref}
\usepackage[alphabetic, lite]{amsrefs} % for bibliography
\begin{document}
\thispagestyle{fancy}
% \subsection*{Definitions}


%***********************************************
 \subsection*{Theory}
 \begin{namedtheorem}[Polynomial facts]\
   \begin{enumerate}
     \item Suppose $f(x)=\anpoly$ with $a_n\ne 0$. We call $n$ the {\bf degree} of $f$, denoted $\deg f$.
     \item A polynomial of degree $n$ has at most $n$ distinct roots.
     \item {\em Equating coefficients}. Given polynomials $f(x)=\anpoly$ and $g(x)=\varpoly{b}{n}$ we have
     \[
     f(x)=g(x) \iff n=m \text{ and } a_i=b_i \text{ for all $i$.}
     \]
     \item A nonzero polynomial is {\bf irreducible} if it cannot be factored into two polynomials of smaller degree. If $f(x)$ is an irreducible polynomial with real coefficients, then $\deg f=1$ or $\deg f=2$.
   \end{enumerate}

 \end{namedtheorem}
\begin{namedtheorem}[Partial fraction decomposition] Let $f(x)/g(x)$ be a rational function (i.e, $f(x)$ and $g(x)$ are both polynomials), and suppose that $\boxed{\deg f<\deg g}$.
\begin{itemize}
  \item If $g(x)$ factors into non-repeated linear factors as
  \[
  g(x)=D(x-a_1)(x-a_2)\cdots (x-a_r),
  \]
  then there is a unique choice of constants $A_1, A_2,\dots, A_r$ such that
  \[
  \frac{f(x)}{g(x)}=\frac{A_1}{x-a_1}+\frac{A_2}{x-a_2}+\cdots +\frac{A_r}{x-a_r}.
  \]
  \item If $g(x)$ factors into non-repeated irreducible linear and quadratic factors as
  \[
  g(x)=D(x-a_1)(x-a_2)\cdots (x-a_r)(x^2+b_1x+c_1)(x^2+b_2x+c_2)\cdots (x^2+b_sx+c_s),
  \]
  there there is a unique choice of constants $A_1,A_2,\dots, A_r, B_1,B_2,\dots, B_s, C_1, C_2,\dots, C_s$ such that
  \[
  \frac{f(x)}{g(x)}=\frac{A_1}{x-a_1}+\frac{A_2}{x-a_2}+\cdots +\frac{A_r}{x-a_r}+\frac{B_1x+C_1}{x^2+b_1x+c_1}+\frac{B_2x+C_2}{x^2+b_2x+c_2}+\cdots +\frac{B_sx+C_s}{x^2+b_sx+c_s}.
  \]
\end{itemize}
\end{namedtheorem}
\begin{comment}
\
\begin{enumerate}
  \item If $\deg f\geq \deg g$, then we can perform long polynomial division to write
  \[
  f(x)=h(x)+\frac{r(x)}{g(x)},
  \]
  where $h(x)$ is a polynomial and $\deg r(x)<\deg g(x)$, and then apply partial fraction decomposition to $r(x)/g(x)$.
  \item There is a more general statement of partial fraction decomposition covering the case where $g(x)$ has repeated irreducible linear and quadratic factors, but we will not use it. See the text if you are interested.
\end{enumerate}
\end{comment}





%***************************************

\subsection*{Procedures}
\begin{namedtheorem}[Partial fraction decomposition]
  Let $f(x)/g(x)$ be a quotient of polynomials, and suppose $\deg f<\deg g$. To compute the partial fraction decomposition of $f(x)/g(x)$ proceed as follows.
  \begin{enumerate}
    \item Factor $g(x)$ into powers of distinct irreducible polynomials.

    {\bf Factoring trick}. If $g(x)$ has integer coefficients and a leading coefficient equal to 1, then any integer roots of $g(x)$ will be factors of the constant term.
    \item Set up the partial fraction decomposition equation with as yet unknown constants ($A_i$, $B_i$, etc.). Clear the denominators of both sides of the equation, resulting in an identity between two polynomials. The polynomial on the right will be expressed in terms of the unknowns ($A_i$, $B_i$, etc.).
    \item To solve for the undetermined constants ($A_i$, $B_i$, etc.) set up and solve a linear system of equations using one of the following techniques.
    \begin{enumerate}
      \item {\em Equate coefficients}. Express the polynomial on right in ``standard form" by collecting like terms. For the left and right polynomials to be equal, their corresponding coefficients must all be equal. This yields a system of equations in the unknowns ($A_i$, $B_i$, etc.) that you must now solve.
      \item {\em Evaluate equality at choices of $x$}. Evaluate the polynomial equation at various explicit choices of $x$. Each evaluation at a specific $x=c$ yields a new linear equation in the unknowns ($A_i$, $B_i$, etc.). Do this enough times so that your system of equations determines the unknowns uniquely. As far as possible, make judicious choices for $x$ to make your algebra easier.
    \end{enumerate}
  \end{enumerate}

\end{namedtheorem}

%*********************************************************
\subsection*{Examples}
\begin{enumerate}
  \item Compute $\displaystyle\int \frac{x+2}{x^2-1}\, dx$
  \item Compute $\displaystyle\int\frac{1}{x^4+3x^2+2}$
  \item Compute $\displaystyle\int\frac{x^2+1}{x^3+2x^2-x-2}\, dx$
  \item Compute $\displaystyle\int \frac{x^2}{x^2+2x-1}\, dx$
  % \item Compute $\displaystyle\int\frac{1}{x^4-5x^2+4}\, dx$
\end{enumerate}




\end{document}
