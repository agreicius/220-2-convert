\documentclass[11pt]{article}
%%%%%%%%%%PACKAGES%%%%%%%%%%%%%%%%%%%%%%%%%%%%%%%%%%%
\usepackage{latexsym}
\usepackage{amssymb, amsmath, amsthm, amsfonts}
\usepackage{stmaryrd} %For \mapsfrom
%\usepackage[fleqn]{amsmath}  % fleqn option makes aligned equations flushed left!
%\usepackage[english]{babel}
%\usepackage{pgf}
\usepackage{mathtools}
\usepackage[mathscr]{eucal}
\usepackage{fancyhdr}
\usepackage{multicol,parcolumns}
\usepackage{enumerate}
%\usepackage{enumitem}
\usepackage[shortlabels]{enumitem}
\usepackage{graphicx}
\usepackage{extarrows}
\usepackage{cancel}
%\usepackage{tikz}
%\usepackage[all,cmtip]{xy} %\SelectTips{cm}{10}
\usepackage[all]{xy} \SelectTips{cm}{10}
%\usepackage{listings} %For code blocks

\input{LatexPreamble}
%%%%%%%%FANCY HEADER%%%%%%%%%
\pagestyle{plain}
\setlength{\headheight}{13.6pt}
\fancyhfoffset[L]{.5in}
%\lhead{\Large \bf{Name:}}
\chead{Executive summary: the natural logarithm function}
\rhead{Math 220-2}
%\lfoot{TURN OVER!}
%\rfoot{TURN OVER!}

%%%%%%%PAGE LAYOUT%%%%%%%%%%%%%
\setlength{\textwidth}{6.5in}
\setlength{\textheight}{9in}

%\setlength{\topmargin}{-.8in}
%\setlength{\columnsep}{1.5in}
\addtolength{\hoffset}{-1 in}
\addtolength{\voffset}{-.5 in}


%%%%%%%THEOREM ENVIRONMENTS%%%%%%%%
\theoremstyle{definition}
\newtheorem*{definition}{Definition}
\newtheorem*{definitions}{Definitions}
\newtheorem*{notation}{Notation}
\newtheorem*{example}{Example}
\newtheorem*{comment}{Comment}
\newtheorem*{comments}{Comments}
\newtheorem*{examples}{Examples}
\newtheorem*{warning}{Warning}
\newtheorem*{theorem}{Theorem}
\newtheorem*{corollary}{Corollary}
\newtheorem*{proposition}{Proposition}
\newtheorem*{lemma}{Lemma}

\newtheoremstyle{named}{}{}{}{}{\bfseries}{.}{.5em}{\thmnote{#3}}
\theoremstyle{named}
\newtheorem*{namedtheorem}{Theorem}

\newcounter{myalgctr}
\newenvironment{myalg}{%      define a custom environment
\bigskip\noindent%         create a vertical offset to previous material
\refstepcounter{myalgctr}% increment the environment's counter
\textbf{Algorithm \themyalgctr}% or \textbf, \textit, ...
\newline%
}{\par\bigskip}  %
\numberwithin{myalgctr}{section}



\newenvironment{solution}{\begin{proof}[Solution]}{\end{proof}}


%%%%%%%%%%HYPERREFS PACKAGE%%%%%%%%%%%%%%%%%
\usepackage[colorlinks]{hyperref}
%\definecolor{webcolor}{rgb}{0.8,0,0.2}
%\definecolor{webbrown}{rgb}{.6,0,0}
%\usepackage[
%        colorlinks,
%       linkcolor=webbrown,  filecolor=webcolor,  citecolor=webbrown,
%        backref
%]{hyperref}
\usepackage[alphabetic, lite]{amsrefs} % for bibliography
\begin{document}
\thispagestyle{fancy}
\subsection*{Definitions}
\begin{namedtheorem}[Natural logarithm] The {\bf natural logarithm function}  is defined as
  \[
  \ln x=\int_1^x\frac{1}{t}\, dt
  \]
  where $x$ is an element of $(0,\infty)$.

\end{namedtheorem}
\begin{namedtheorem}[Euler's number] {\bf Euler's number}, denoted $e$, is the unique number satisfying $\ln e=1$. In other words, $e$ is the number satisfying
  \[
  1=\int_1^e\frac{1}{t}\, dt.
  \]

\end{namedtheorem}
%***********************************************
\subsection*{Theory}
\begin{namedtheorem}[Properties of the natural logarithm]
  The following properties hold:
  \begin{enumerate}
    \item The natural logarithm is differentiable (hence also continuous) on $(0,\infty)$ and satisfies
    \[
    \frac{d}{dx}\ln x=\frac{1}{x}.
    \]
    for all $x$ in $(0,\infty)$.
    \item The natural logarithm is increasing on $(0,\infty)$ and hence one-to-one. The graph of $\ln$ is always concave down.
    \item We have
    \begin{align*}
      \lim_{x\to\infty}\ln x&=\infty\\
      \lim_{x\to 0^+}\ln x&=-\infty
    \end{align*}
    \item The domain of $\ln$ is $(0,\infty)$; the range of $\ln$ is $(-\infty, \infty)$.
    \item $\ln 1=0$.
    \item We have
    \begin{align*}
      \ln(ab)&=\ln a+\ln b,  \text{ for all $a,b\in (0,\infty)$.}\\
      \ln(a/b)&=\ln a-\ln b,  \text{ for all $a,b\in (0,\infty)$, $b\ne 0$.}\\
      \ln a^r&=r\ln a,  \text{ for all $a\in (0,\infty)$ and $r$ rational.}
    \end{align*}
  \end{enumerate}
\end{namedtheorem}

\begin{corollary}
  The function $f(x)=\ln\lvert x\rvert$ is an antiderivative of $1/x$ on its entire domain $D=(-\infty, 0)\cup (0,\infty)$: i.e., we have
  \[
  \int\frac{1}{x}\, dx=\ln \vert x\rvert\, dx+C.
  \]
\end{corollary}
\begin{namedtheorem}[Further trigonometric antiderivative formulas]
  The following antiderivative formulas hold:
  \begin{enumerate}
    \item $\displaystyle\int \tan x\, dx=-\ln\lvert \cos x\rvert+C=\ln\lvert\sec x\rvert+C$
    \item $\displaystyle\int \cot x\, dx=\ln\vert \sin x\vert+C$
    \item $\displaystyle\int \sec x\, dx=\ln\vert \sec x+\tan x\vert+C$
    \item $\displaystyle\int \csc x\, dx=-\ln\vert \csc x+\cot x\vert+C$
  \end{enumerate}

\end{namedtheorem}


%***************************************

% \subsection*{Procedures}
%
% %*********************************************************
% \subsection*{Examples}





\end{document}
