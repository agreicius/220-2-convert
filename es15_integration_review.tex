\documentclass[11pt]{article}
%%%%%%%%%%PACKAGES%%%%%%%%%%%%%%%%%%%%%%%%%%%%%%%%%%%
\usepackage{latexsym}
\usepackage{amssymb, amsmath, amsthm, amsfonts}
\usepackage{stmaryrd} %For \mapsfrom
%\usepackage[fleqn]{amsmath}  % fleqn option makes aligned equations flushed left!
%\usepackage[english]{babel}
%\usepackage{pgf}
\usepackage{mathtools}
\usepackage[mathscr]{eucal}
\usepackage{fancyhdr}
\usepackage{multicol,parcolumns}
\usepackage{enumerate}
%\usepackage{enumitem}
\usepackage[shortlabels]{enumitem}
\usepackage{graphicx}
\usepackage{extarrows}
\usepackage{cancel}
%\usepackage{tikz}
%\usepackage[all,cmtip]{xy} %\SelectTips{cm}{10}
\usepackage[all]{xy} \SelectTips{cm}{10}
%\usepackage{listings} %For code blocks

\input{LatexPreamble}
%%%%%%%%FANCY HEADER%%%%%%%%%
\pagestyle{plain}
\setlength{\headheight}{13.6pt}
\fancyhfoffset[L]{.5in}
%\lhead{\Large \bf{Name:}}
\chead{Executive summary: taking stock of integration techniques}
\rhead{Math 220-2}
%\lfoot{TURN OVER!}
%\rfoot{TURN OVER!}

%%%%%%%PAGE LAYOUT%%%%%%%%%%%%%
\setlength{\textwidth}{6.5in}
\setlength{\textheight}{9in}

%\setlength{\topmargin}{-.8in}
%\setlength{\columnsep}{1.5in}
\addtolength{\hoffset}{-1 in}
\addtolength{\voffset}{-.5 in}


%%%%%%%THEOREM ENVIRONMENTS%%%%%%%%
\theoremstyle{definition}
\newtheorem*{definition}{Definition}
\newtheorem*{definitions}{Definitions}
\newtheorem*{notation}{Notation}
\newtheorem*{example}{Example}
\newtheorem*{comment}{Comment}
\newtheorem*{comments}{Comments}
\newtheorem*{examples}{Examples}
\newtheorem*{warning}{Warning}
\newtheorem*{theorem}{Theorem}
\newtheorem*{corollary}{Corollary}
\newtheorem*{proposition}{Proposition}
\newtheorem*{lemma}{Lemma}

\newtheoremstyle{named}{}{}{}{}{\bfseries}{.}{.5em}{\thmnote{#3}}
\theoremstyle{named}
\newtheorem*{namedtheorem}{Theorem}

\newcounter{myalgctr}
\newenvironment{myalg}{%      define a custom environment
   \bigskip\noindent%         create a vertical offset to previous material
   \refstepcounter{myalgctr}% increment the environment's counter
   \textbf{Algorithm \themyalgctr}% or \textbf, \textit, ...
   \newline%
   }{\par\bigskip}  %
\numberwithin{myalgctr}{section}



\newenvironment{solution}{\begin{proof}[Solution]}{\end{proof}}


%%%%%%%%%%HYPERREFS PACKAGE%%%%%%%%%%%%%%%%%
\usepackage[colorlinks]{hyperref}
%\definecolor{webcolor}{rgb}{0.8,0,0.2}
%\definecolor{webbrown}{rgb}{.6,0,0}
%\usepackage[
%        colorlinks,
%       linkcolor=webbrown,  filecolor=webcolor,  citecolor=webbrown,
%        backref
%]{hyperref}
\usepackage[alphabetic, lite]{amsrefs} % for bibliography
\begin{document}
\thispagestyle{fancy}
Having introduced a wealth of new derivative/integral formulas and rules, we now take a moment to give an overview of our integration techniques, and apply them in combination with some algebraic methods to solving integrals in the wild.

%***********************************************
 \subsection*{Theory}
\begin{namedtheorem}[Derivative/antiderivative formula compendium] We collect here our various derivative/antiderivative formulas.
\begin{align*}
  \frac{d}{dx} x^r=rx^{r-1}& \iff \int x^r \, dx=\frac{1}{r+1}x^{r+1}+C, r\ne -1\\
  \frac{d}{dx} \sin x=\cos x& \iff \int\cos x \, dx=\sin x+C\\
  \frac{d}{dx}\cos x=-\sin x & \iff \int \sin x \, dx=-\cos x+C\\
  \frac{d}{dx}\tan x=\sec^2x & \iff \int \sec^2 x \, dx=\tan x+C\\
  \frac{d}{dx}\cot x=-\csc^2x & \iff \int \csc ^2x\, dx=-\cot x+C\\
  \frac{d}{dx}\sec x=\sec x\tan x & \iff \int \sec x\tan x \, dx=\sec x+C\\
  \frac{d}{dx}\csc x=-\csc x\cot x & \iff \int\csc x\cot x \, dx=-\csc x+C\\
  \frac{d}{dx}\ln\vert x\vert=\frac{1}{x} & \iff \int\frac{1}{x} \, dx=\ln\vert x\vert+C \\
  \frac{d}{dx}\ln\vert \cos x \vert=-\tan x & \iff \int\tan x \, dx=-\ln\vert \cos x \vert+C=\ln\vert\sec x\vert+C \\
  \frac{d}{dx}\ln\vert \sin x\vert=\cot x & \iff \int\cot x \, dx=\ln\vert \sin x\vert+C \\
  \frac{d}{dx}\ln\vert \sec x+\tan x\vert=\sec x & \iff \int\sec x \, dx=\ln\vert \sec x+\tan x\vert+C \\
  \frac{d}{dx}\ln\vert \csc x+\cot x\vert=-\csc x & \iff \int\csc x \, dx=-\ln\vert \csc x+\cot x\vert+C\\
  \frac{d}{dx}e^x=e^x & \iff \int e^x \, dx=e^x+C\\
  \frac{d}{dx}a^x=(\ln a)a^x & \iff \int a^x \, dx=\frac{1}{\ln a}a^x+C\\
  \frac{d}{dx}\log_a \vert x\vert =\frac{1}{(\ln a)\, x} & \iff \int \frac{1}{(\ln a)\, x} \, dx=\log_a\vert x\vert+C\\
  \frac{d}{dx} \arcsin x=\frac{1}{\sqrt{1-x^2}}&\iff \int \frac{1}{\sqrt{1-x^2}}\, dx=\arcsin x+C \\
  \frac{d}{dx} \arccos x=-\frac{1}{\sqrt{1-x^2}}&\iff \int \frac{1}{\sqrt{1-x^2}}\, dx=-\arccos x+C \\
  \frac{d}{dx} \arctan x=\frac{1}{1+x^2}&\iff \int \frac{1}{1+x^2}\, dx=\arctan x+C .
\end{align*}

\end{namedtheorem}

\newpage
%***************************************



%*********************************************************
\subsection*{Examples}
Each of the integral computations below will combine various integral formulas, substitution, and an algebraic method.
\begin{enumerate}
  \item Compute $\displaystyle \int \frac{1}{x^2-6x+18}$
  \item Compute $\displaystyle \int \frac{4x^3+3x+1}{4x^2+1}\, dx$

  {\bf Hint}. Use polynomial division with remainder.
  \item Compute $\displaystyle\int \frac{1}{e^x+e^{-x}}\, dx$
  \item Compute $\displaystyle \frac{3}{\sqrt{e^{2x}-2}}\, dx$

  \item Compute $\displaystyle \int_0^{\pi/3}\sin^2(2x)\cos(3x) \, dx$

  {\bf Hint}. Make use of some of the following product-to-sum identities.
  \begin{align}
    \cos\theta\cos\phi&=\frac{\cos(\theta-\phi)+\cos(\theta+\phi)}{2} \\
    \sin\theta\sin\phi&=\frac{\cos(\theta-\phi)-\cos(\theta+\phi)}{2} \\
    \sin\theta\cos\phi&= \frac{\sin(\theta-\phi)+\sin(\theta+\phi)}{2}\\
    \cos^2\theta&=\frac{1+\cos2\theta}{2}\\
    \sin^2\theta&=\frac{1-\cos2\theta}{2}
  \end{align}

\end{enumerate}




\end{document}
