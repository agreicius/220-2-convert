\documentclass[11pt]{article}
%%%%%%%%%%PACKAGES%%%%%%%%%%%%%%%%%%%%%%%%%%%%%%%%%%%
\usepackage{latexsym}
\usepackage{amssymb, amsmath, amsthm, amsfonts}
\usepackage{stmaryrd} %For \mapsfrom
%\usepackage[fleqn]{amsmath}  % fleqn option makes aligned equations flushed left!
%\usepackage[english]{babel}
%\usepackage{pgf}
\usepackage{mathtools}
\usepackage[mathscr]{eucal}
\usepackage{fancyhdr}
\usepackage{multicol,parcolumns}
\usepackage{enumerate}
%\usepackage{enumitem}
\usepackage[shortlabels]{enumitem}
\usepackage{graphicx}
\usepackage{extarrows}
\usepackage{cancel}
%\usepackage{tikz}
%\usepackage[all,cmtip]{xy} %\SelectTips{cm}{10}
\usepackage[all]{xy} \SelectTips{cm}{10}
%\usepackage{listings} %For code blocks

\input{LatexPreamble}
%%%%%%%%FANCY HEADER%%%%%%%%%
\pagestyle{plain}
\setlength{\headheight}{13.6pt}
\fancyhfoffset[L]{.5in}
%\lhead{\Large \bf{Name:}}
\chead{Executive summary: trigonometric integrals}
\rhead{Math 220-2}
%\lfoot{TURN OVER!}
%\rfoot{TURN OVER!}

%%%%%%%PAGE LAYOUT%%%%%%%%%%%%%
\setlength{\textwidth}{6.5in}
\setlength{\textheight}{9in}

%\setlength{\topmargin}{-.8in}
%\setlength{\columnsep}{1.5in}
\addtolength{\hoffset}{-1 in}
\addtolength{\voffset}{-.5 in}


%%%%%%%THEOREM ENVIRONMENTS%%%%%%%%
\theoremstyle{definition}
\newtheorem*{definition}{Definition}
\newtheorem*{definitions}{Definitions}
\newtheorem*{notation}{Notation}
\newtheorem*{example}{Example}
\newtheorem*{comment}{Comment}
\newtheorem*{comments}{Comments}
\newtheorem*{examples}{Examples}
\newtheorem*{warning}{Warning}
\newtheorem*{theorem}{Theorem}
\newtheorem*{corollary}{Corollary}
\newtheorem*{proposition}{Proposition}
\newtheorem*{lemma}{Lemma}

\newtheoremstyle{named}{}{}{}{}{\bfseries}{.}{.5em}{\thmnote{#3}}
\theoremstyle{named}
\newtheorem*{namedtheorem}{Theorem}

\newcounter{myalgctr}
\newenvironment{myalg}{%      define a custom environment
\bigskip\noindent%         create a vertical offset to previous material
\refstepcounter{myalgctr}% increment the environment's counter
\textbf{Algorithm \themyalgctr}% or \textbf, \textit, ...
\newline%
}{\par\bigskip}  %
\numberwithin{myalgctr}{section}



\newenvironment{solution}{\begin{proof}[Solution]}{\end{proof}}


%%%%%%%%%%HYPERREFS PACKAGE%%%%%%%%%%%%%%%%%
\usepackage[colorlinks]{hyperref}
%\definecolor{webcolor}{rgb}{0.8,0,0.2}
%\definecolor{webbrown}{rgb}{.6,0,0}
%\usepackage[
%        colorlinks,
%       linkcolor=webbrown,  filecolor=webcolor,  citecolor=webbrown,
%        backref
%]{hyperref}
\usepackage[alphabetic, lite]{amsrefs} % for bibliography
\begin{document}
\thispagestyle{fancy}
\subsection*{Definitions}


%***********************************************
\subsection*{Theory}
\begin{namedtheorem}[Trigonometric identities] The following identities hold for all $\theta, \phi\in\mathbb{R}$.
  \begin{multicols}{2}
    \begin{enumerate}
      \item $\displaystyle\cos\theta\cos\phi=\frac{\cos(\theta-\phi)+\cos(\theta+\phi)}{2}$
      \item $\displaystyle\sin\theta\sin\phi=\frac{\cos(\theta-\phi)-\cos(\theta+\phi)}{2}$
      \item $\displaystyle\sin\theta\cos\phi= \frac{\sin(\theta-\phi)+\sin(\theta+\phi)}{2}$
      \item $\displaystyle\cos^2\theta=\frac{1+\cos2\theta}{2}$
      \item $\displaystyle\sin^2\theta=\frac{1-\cos2\theta}{2}$
    \end{enumerate}
  \end{multicols}

\end{namedtheorem}


%***************************************

\subsection*{Procedures}
\begin{comment}
The basic strategy for computing integrals of functions of the form $\sin^m x\cos^n x$ or $\tan^m x\sec^n x$ is to use one of the four substitutions
\begin{align*}
  u&=\sin x & u&=\cos x & u&=\tan x & u&=\sec x \\
  du&=\cos x\, dx & du&=-\sin x\, dx & du&=\sec^2 x\, dx & du&=\sec x\tan x\, dx,
\end{align*}
``peel off" what is necessary for $du$, and express the rest of the integrand as a polynomial in $u$ using the trigonometric identities.
\begin{align*}
\sin^2 x+\cos^2 x&=1 & \sec^2 x&=\tan^2+1.
\end{align*}
\end{comment}
\begin{namedtheorem}[Integrating $\sin^m x\cos^n x$] Let $m$ and $n$ be nonnegative integers. When computing
  \[
  \int \sin^m x\cos^n x\, dx
  \]
  the following strategies often help.
  \begin{enumerate}
    \item If $m=2k+1$ is odd, write
    \[
    \int \sin^m x\cos^n x\, dx=\int (1-\cos^2x)^k\cos^n x\sin x\, dx
    \]
    and use the substitution $u=\sin x, du=\cos x\, dx$.
    \item If $n=2k+1$ is odd, write
    \[
    \int \sin^m x\cos^n x\, dx=\int \sin^m x(1-\sin^2x)^k\cos x \, dx
    \]
    and use the substitution $u=\cos x, du=-\sin x\, dx$.
    \item If $m$ and $n$ are both even use $\displaystyle\sin^2 x=\frac{1-\cos 2x}{2}$ and $\displaystyle\cos^2 x=\frac{1+\cos 2x}{2}$ to reduce to a lower power of $\cos 2x$.
  \end{enumerate}

\end{namedtheorem}
\newpage
\begin{namedtheorem}[Integrating $\tan^m x\sec^n x$] Let $m$ and $n$ be nonnegative integers. When computing
  \[
  \int \tan^m x\sec^n x \, dx
  \]
  the following strategies often help.
  \begin{enumerate}
    \item If $m=2k+1$ is odd and $n\geq 1$, write
    \[
    \int \tan^m x\sec^n x \, dx=\int (\sec^2 x-1)^k\sec^{n-1} x\sec x\tan x \, dx
    \]
    and use the substitution $u=\sec x, du=\sec x\tan x\, dx$.
    \item If $n=2k$ is even, write
    \[
    \int \tan^m x\sec^n x \, dx=\int (\tan^2 x+1)^{k-1}\tan^m x\sec^2 x\, dx
    \]
    and use the substitution $u=\tan x, du=\sec^2 x\, dx$.

    \item If $m$ is even and $n$ is odd, express everything in terms of $\sec x$ and possibly use integration by parts.
  \end{enumerate}

\end{namedtheorem}

%*********************************************************
\subsection*{Examples}
Compute the following indefinite integrals.
\begin{enumerate}
  \item $\displaystyle \int \sin^3x \cos^2 x\, dx$
  \item $\displaystyle \int \sin^2 x\cos^4 x\, dx$
  \item $\displaystyle \int \sec^4 x\, dx$
  \item $\displaystyle \int \tan^5 x\sec^7 x\, dx$
  \item $\displaystyle \int \sec^3 x\, dx$
  \item $\displaystyle \int \tan^5 x \, dx$

\end{enumerate}




\end{document}
