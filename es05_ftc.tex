\documentclass[11pt]{article}
%%%%%%%%%%PACKAGES%%%%%%%%%%%%%%%%%%%%%%%%%%%%%%%%%%%
\usepackage{latexsym}
\usepackage{amssymb, amsmath, amsthm, amsfonts}
\usepackage{stmaryrd} %For \mapsfrom
%\usepackage[fleqn]{amsmath}  % fleqn option makes aligned equations flushed left!
%\usepackage[english]{babel}
%\usepackage{pgf}
\usepackage{mathtools}
\usepackage[mathscr]{eucal}
\usepackage{fancyhdr}
\usepackage{multicol,parcolumns}
\usepackage{enumerate}
%\usepackage{enumitem}
\usepackage[shortlabels]{enumitem}
\usepackage{graphicx}
\usepackage{extarrows}
\usepackage{cancel}
%\usepackage{tikz}
%\usepackage[all,cmtip]{xy} %\SelectTips{cm}{10}
\usepackage[all]{xy} \SelectTips{cm}{10}
%\usepackage{listings} %For code blocks

\input{LatexPreamble}
%%%%%%%%FANCY HEADER%%%%%%%%%
\pagestyle{plain}
\setlength{\headheight}{13.6pt}
\fancyhfoffset[L]{.5in}
%\lhead{\Large \bf{Name:}}
\chead{Executive summary: fundamental theorem of calculus}
\rhead{Math 220-2}
%\lfoot{TURN OVER!}
%\rfoot{TURN OVER!}

%%%%%%%PAGE LAYOUT%%%%%%%%%%%%%
\setlength{\textwidth}{6.5in}
\setlength{\textheight}{9in}

%\setlength{\topmargin}{-.8in}
%\setlength{\columnsep}{1.5in}
\addtolength{\hoffset}{-1 in}
\addtolength{\voffset}{-.5 in}


%%%%%%%THEOREM ENVIRONMENTS%%%%%%%%
\theoremstyle{definition}
\newtheorem*{definition}{Definition}
\newtheorem*{definitions}{Definitions}
\newtheorem*{notation}{Notation}
\newtheorem*{example}{Example}
\newtheorem*{comment}{Comment}
\newtheorem*{comments}{Comments}
\newtheorem*{examples}{Examples}
\newtheorem*{warning}{Warning}
\newtheorem*{theorem}{Theorem}
\newtheorem*{corollary}{Corollary}
\newtheorem*{proposition}{Proposition}
\newtheorem*{lemma}{Lemma}

\newtheoremstyle{named}{}{}{}{}{\bfseries}{.}{.5em}{\thmnote{#3}}
\theoremstyle{named}
\newtheorem*{namedtheorem}{Theorem}

\newcounter{myalgctr}
\newenvironment{myalg}{%      define a custom environment
   \bigskip\noindent%         create a vertical offset to previous material
   \refstepcounter{myalgctr}% increment the environment's counter
   \textbf{Algorithm \themyalgctr}% or \textbf, \textit, ...
   \newline%
   }{\par\bigskip}  %
\numberwithin{myalgctr}{section}



\newenvironment{solution}{\begin{proof}[Solution]}{\end{proof}}


%%%%%%%%%%HYPERREFS PACKAGE%%%%%%%%%%%%%%%%%
\usepackage[colorlinks]{hyperref}
%\definecolor{webcolor}{rgb}{0.8,0,0.2}
%\definecolor{webbrown}{rgb}{.6,0,0}
%\usepackage[
%        colorlinks,
%       linkcolor=webbrown,  filecolor=webcolor,  citecolor=webbrown,
%        backref
%]{hyperref}
\usepackage[alphabetic, lite]{amsrefs} % for bibliography
\begin{document}
\thispagestyle{fancy}
\subsection*{Definitions}
\begin{namedtheorem}[Average value of a function] Let $f$ be integrable on $[a,b]$. The {\bf average value of $f$ over $[a,b]$} is defined as
  \[
  \frac{1}{b-a}\int_a^bf(x)\, dx.
  \]
\end{namedtheorem}
\begin{namedtheorem}[Difference-evaluation notation] Let $g$ be a real-valued function, and let $a,b$ be elements of the domain of $g$. The notation $\Bigl[g(x)\Bigr]_a^b$  is defined as follows:
  \[
  \Bigl[ g(x)\Bigr ]_a^b=g(b)-g(a).
  \]
  It is worthwile recording some simple identities involving this notation:
  \begin{align*}
    \Bigl[ f(x)\pm g(x)\Bigr ]_a^b&=\Bigl[f(x)\Bigr ]_a^b\pm \Bigl[ g(x)\Bigr ]_a^b\\
    \Bigl[ cg(x)\Bigr]_a^b&=c\Bigl[g(x)\Bigr]_a^b
  \end{align*}
  We will often abbreviate the notation $\Bigl[g(x)\Bigr]_a^b$ to $g(x)\Bigr]_a^b$.
\end{namedtheorem}

%***********************************************
 \subsection*{Theory}
\begin{namedtheorem}[Fundamental theorem of calculus (I)] Let $f$ be continuous on an open interval $I$ containing $a$. Let $F(x)$ be the function defined on $I$ as
  \[
  F(x)=\int_a^x f(t)\, dt.
  \]
Then $F(x)$ is differentiable on $I$ and
\[
F'(x)=\frac{d}{dx}\int_a^xf(t)\, dt=f(x)
\]
for all $x\in I$.
\end{namedtheorem}
\begin{corollary}
  If $f$ is continuous on the open interval $I$, then $f$ has an antiderivative on $I$.
\end{corollary}
\begin{namedtheorem}[Fundamental theorem of calculus (II)] Let $f$ be continuous on the interval $[a,b]$. If $F(x)$ is an antiderivative of $f(x)$ on $[a,b]$, then
  \[
  \int_a^b f(x)\, dx=F(b)-F(a).
  \]

\end{namedtheorem}
\begin{namedtheorem}[FTC II: rate of change version] Suppose $g$ is differentiable on $[a,b]$. The derivative function $g'$ computes the (instantaneous) rate of change of $g$ with respect to $x$. By FTC II, we have:
  \[
  \int_a^b g'(x)\, dx=g(b)-g(a).
  \]
  In other words, the integral of the {\em rate of change} of a function over $[a,b]$ is the {\em net change} of that function from $a$ to $b$.

\end{namedtheorem}
%***************************************

\subsection*{Procedures}
\begin{namedtheorem}[Antiderivative method of computing integrals] Suppose $f(x)$ is continuous on $[a,b]$. The antiderivative method for computing $\int_a^b f(x)\, dx$ proceeds as follows:
  \begin{enumerate}
    \item Find an antiderivative of $f$: i.e., find $F$ such that $F'(x)=f(x)$ for all $x\in [a,b]$.
    \item By the fundamental theorem of calculus (II) we have
    \[
      \int_a^bf(x)\, dx=F(b)-F(a).
    \]
  \end{enumerate}

\end{namedtheorem}

%*********************************************************
\subsection*{Examples}
\begin{enumerate}
  \item Use the fundamental theorem of calculus to compute the following definite integrals.
  \begin{enumerate}
    \item $\ds\int_a^b 1-x^3\, dx$
    \item $\ds\int_0^{10} \frac{1}{\sqrt{2t+1}}\, dt$
    \item $\ds\int_{3\pi/4}^{\pi}\sec^2 u\, du$.
  \end{enumerate}
  \item Let $f(x)=\sin(x/3)$, and let $\mathcal{C}$ be the graph of $f$.

  For each region $\mathcal{R}$ compute the area of $\mathcal{R}$ and the signed area of $\mathcal{R}$.

  Include a diagram of $\mathcal{C}$ and $\mathcal{R}$. Make sure your answer is consistent with your graph. If your answer happens to be 0, use the diagram to explain why.
  \begin{enumerate}
    \item $\mathcal{R}$ is the region between $\mathcal{C}$ and the $x$-axis, from $x=-\pi$ to $x=\pi$.
    \item $\mathcal{R}$ is the region between $\mathcal{C}$ and the $x$-axis, from $x=-\pi$ to $x=\pi/2$.
    \item $\mathcal{R}$ is the region between $\mathcal{C}$ and the $x$-axis, from $x=0$ to $x=6\pi$.
  \end{enumerate}
  \item Let $F(x)=\int_1^x \frac{1}{t^2}\, dt$. Make a table of values of $F(x)$ for $x=1, 2, 3, 4, 5$. Explain graphically what $F(b)$ is for any $b\geq 1$.
  \item For each $F(x)$ defined below, use the fundamental theorem of calculus (along with some other useful pieces of theory) to compute $\ds F'(x)=\frac{d}{dx}F(x)$.
  \begin{enumerate}
    \item $\ds F(x)=\int_{x}^5 \sqrt{t+1}\, dt $
    \item $\ds F(x)=\int_{-2}^{\sin x}\cos(u^2) \, du$
    \item $\ds F(x)=\int_{4x}^{\sqrt{x^2+1}}\sin(s^2)\, ds$

  \end{enumerate}
\end{enumerate}




\end{document}
