\documentclass[11pt]{article}
%%%%%%%%%%PACKAGES%%%%%%%%%%%%%%%%%%%%%%%%%%%%%%%%%%%
\usepackage{latexsym}
\usepackage{amssymb, amsmath, amsthm, amsfonts}
\usepackage{stmaryrd} %For \mapsfrom
%\usepackage[fleqn]{amsmath}  % fleqn option makes aligned equations flushed left!
%\usepackage[english]{babel}
%\usepackage{pgf}
\usepackage{mathtools}
\usepackage[mathscr]{eucal}
\usepackage{fancyhdr}
\usepackage{multicol,parcolumns}
\usepackage{enumerate}
%\usepackage{enumitem}
\usepackage[shortlabels]{enumitem}
\usepackage{graphicx}
\usepackage{extarrows}
\usepackage{cancel}
%\usepackage{tikz}
%\usepackage[all,cmtip]{xy} %\SelectTips{cm}{10}
\usepackage[all]{xy} \SelectTips{cm}{10}
%\usepackage{listings} %For code blocks

\input{LatexPreamble}
%%%%%%%%FANCY HEADER%%%%%%%%%
\pagestyle{plain}
\setlength{\headheight}{13.6pt}
\fancyhfoffset[L]{.5in}
%\lhead{\Large \bf{Name:}}
\chead{Executive summary: Riemann sums}
\rhead{Math 220-2}
%\lfoot{TURN OVER!}
%\rfoot{TURN OVER!}

%%%%%%%PAGE LAYOUT%%%%%%%%%%%%%
\setlength{\textwidth}{6.5in}
\setlength{\textheight}{9in}

%\setlength{\topmargin}{-.8in}
%\setlength{\columnsep}{1.5in}
\addtolength{\hoffset}{-1 in}
\addtolength{\voffset}{-.5 in}


%%%%%%%THEOREM ENVIRONMENTS%%%%%%%%
\theoremstyle{definition}
\newtheorem*{definition}{Definition}
\newtheorem*{definitions}{Definitions}
\newtheorem*{notation}{Notation}
\newtheorem*{example}{Example}
\newtheorem*{comment}{Comment}
\newtheorem*{comments}{Comments}
\newtheorem*{examples}{Examples}
\newtheorem*{warning}{Warning}
\newtheorem*{theorem}{Theorem}
\newtheorem*{corollary}{Corollary}
\newtheorem*{proposition}{Proposition}
\newtheorem*{lemma}{Lemma}

\newtheoremstyle{named}{}{}{}{}{\bfseries}{.}{.5em}{\thmnote{#3}}
\theoremstyle{named}
\newtheorem*{namedtheorem}{Theorem}

\newcounter{myalgctr}
\newenvironment{myalg}{%      define a custom environment
   \bigskip\noindent%         create a vertical offset to previous material
   \refstepcounter{myalgctr}% increment the environment's counter
   \textbf{Algorithm \themyalgctr}% or \textbf, \textit, ...
   \newline%
   }{\par\bigskip}  %
\numberwithin{myalgctr}{section}



\newenvironment{solution}{\begin{proof}[Solution]}{\end{proof}}


%%%%%%%%%%HYPERREFS PACKAGE%%%%%%%%%%%%%%%%%
\usepackage[colorlinks]{hyperref}
%\definecolor{webcolor}{rgb}{0.8,0,0.2}
%\definecolor{webbrown}{rgb}{.6,0,0}
%\usepackage[
%        colorlinks,
%       linkcolor=webbrown,  filecolor=webcolor,  citecolor=webbrown,
%        backref
%]{hyperref}
\usepackage[alphabetic, lite]{amsrefs} % for bibliography
\begin{document}
\thispagestyle{fancy}
\subsection*{Definitions}
\begin{namedtheorem}[Sigma notation] Given real numbers $a_1, a_2, \dots, a_n$ the notation
  \[
  \sum_{i=1}^{n}a_i
  \]
denotes their sum: i.e.,
\[
\sum_{i=1}^{n}a_i=a_1+a_2+\cdots +a_n.
\]
More generally given any sequence of real numbers $a_{m}, a_{m+1},\dots$, we define
\[
\sum_{k=m}^na_k=a_m+a_{m+1}+\cdots +a_{n}.
\]
If the terms in the sequence are given by a formula of the form $a_k=f(k)$, then we also write
\[
\sum_{k=m}^nf(k)
\]
for $\ds\sum_{k=m}^na_k$.

\end{namedtheorem}
\begin{namedtheorem}[Riemann sums] Let $f$ be a function defined on the interval $[a,b]$, and let $n$ be a positive integer. A {\bf partition} of $[a,b]$ into $n$ subintervals is a choice of points $x_0, x_1,\dots, x_n$ satisfying
  \[
  a=x_0<x_1<x_2<\cdots <x_n=b.
  \]
Such a partition gives rise to $n$ subintervals of $[a,b]$:
\[
I_1=[x_0, x_1], I_2=[x_1, x_2], \dots , I_n=[x_{n-1}, x_n].
\]
The $k$-th subinterval has length
\[
\Delta x_k=x_{k}-x_{k-1}.
\]
Given a choice of sample points $c_k\in I_k$ for each subinterval, the corresponding {\bf Riemann sum} is
\[
\sum_{k=1}^n f(c_k)\Delta x_k=f(c_1)(x_1-x_0)+f(c_2)(x_2-x_1)+\cdots +f(c_n)(x_n-x_{n-1}).
\]
As with our estimates, we call the Riemann sum a left/right/midpoint/upper/lower sum if the sample points $c_k$ are picked using the corresponding rule. Thus the left Riemann sum corresponding to the partition above is
\[
\sum_{k=1}^nf(x_{k-1})\Delta x_k
\]
and the right Riemann sum is
\[
\sum_{k=1}^nf(x_k)\Delta x_{k}.
\]


\end{namedtheorem}
%***************************************

% \subsection*{Procedures}

%***********************************************
 \subsection*{Theory}
 \begin{namedtheorem}[Summation formulas] Let $n$ be a positive integer. The following summation equalities hold:
   \begin{enumerate}[topsep=0pt, itemsep=0pt]
     \item $\ds\sum_{k=1}^n 1=n$
     \item $\ds \sum_{k=1}^nk=\frac{n(n+1)}{2}$
     \item $\ds\sum_{k=1}^nk^2=\frac{n(n+1)(2n+1)}{6}$
     \item $\ds \sum_{k=1}^nk^3=\left(\frac{n(n+1)}{2}\right)^2$.
   \end{enumerate}

 \end{namedtheorem}
\begin{namedtheorem}[Summation rules]
Given any sequences $a_m, a_{m+1}, \dots$ and $b_m, b_{m+1}, \dots$, and any $c\in \R$, the following equalities hold:
\begin{enumerate}
  \item $\ds\sum_{k=m}^na_k+\sum_{k=m}^nb_k=\sum_{k=m}^n(a_k+b_k)$
  \item $\ds\sum_{k=m}^na_k-\sum_{k=m}^nb_k=\sum_{k=m}^n(a_k-b_k)$
  \item $\ds\sum_{k=m}^nca_k=c\sum_{k=m}^na_k$.
\end{enumerate}

\end{namedtheorem}

%*********************************************************
\subsection*{Examples}
\begin{enumerate}
  \item Let $p_1, p_2,p_3, \dots, $ be the sequence of prime numbers in increasing order: i.e., $p_1=2$, $p_2=3$, $p_3=5$, etc. Compute
  $
  \ds\sum_{k=3}^{6}p_k
  $.
  \item Compute
  $
  \ds \sum_{k=1}^{21}\sin(\pi k/2)
  $
  \item Compute $\ds \sum_{k=1}^{5} 5k^3-10k+2$ using appropriate summation rules and formulas.
  \item Let $n$ be a positive integer and define $R_n$ to be the right Riemann sum of $f(x)=1-x^3$ corresponding to the partition of $[0,1]$ into $n$ equal subintervals.
  \begin{enumerate}
    \item Derive a closed formula for $R_n$. Your answer will be expressed in terms of $n$.
    \item Compute $\lim_{n\to \infty}R_n$. Look familiar?
    \item Now do the same thing with $L_n$, the left Riemann sum of $f(x)$ corresponding to the partion of $[0,1]$ into $n$ equal subintervals.

    {\bf Hint}. For the closed formula of $L_n$ use an ``index-shift" technique to simplify the summation.
  \end{enumerate}
\end{enumerate}



\end{document}
