\documentclass[11pt]{article}
%%%%%%%%%%PACKAGES%%%%%%%%%%%%%%%%%%%%%%%%%%%%%%%%%%%
\usepackage{latexsym}
\usepackage{amssymb, amsmath, amsthm, amsfonts}
\usepackage{stmaryrd} %For \mapsfrom
%\usepackage[fleqn]{amsmath}  % fleqn option makes aligned equations flushed left!
%\usepackage[english]{babel}
%\usepackage{pgf}
\usepackage{mathtools}
\usepackage[mathscr]{eucal}
\usepackage{fancyhdr}
\usepackage{multicol,parcolumns}
\usepackage{enumerate}
%\usepackage{enumitem}
\usepackage[shortlabels]{enumitem}
\usepackage{graphicx}
\usepackage{extarrows}
\usepackage{cancel}
%\usepackage{tikz}
%\usepackage[all,cmtip]{xy} %\SelectTips{cm}{10}
\usepackage[all]{xy} \SelectTips{cm}{10}
%\usepackage{listings} %For code blocks

\input{LatexPreamble}
%%%%%%%%FANCY HEADER%%%%%%%%%
\pagestyle{plain}
\setlength{\headheight}{13.6pt}
\fancyhfoffset[L]{.5in}
%\lhead{\Large \bf{Name:}}
\chead{Executive summary: first-order linear differential equations}
\rhead{Math 220-2}
%\lfoot{TURN OVER!}
%\rfoot{TURN OVER!}

%%%%%%%PAGE LAYOUT%%%%%%%%%%%%%
\setlength{\textwidth}{6.5in}
\setlength{\textheight}{9in}

%\setlength{\topmargin}{-.8in}
%\setlength{\columnsep}{1.5in}
\addtolength{\hoffset}{-1 in}
\addtolength{\voffset}{-.5 in}


%%%%%%%THEOREM ENVIRONMENTS%%%%%%%%
\theoremstyle{definition}
\newtheorem*{definition}{Definition}
\newtheorem*{definitions}{Definitions}
\newtheorem*{notation}{Notation}
\newtheorem*{example}{Example}
\newtheorem*{comment}{Comment}
\newtheorem*{comments}{Comments}
\newtheorem*{examples}{Examples}
\newtheorem*{warning}{Warning}
\newtheorem*{theorem}{Theorem}
\newtheorem*{corollary}{Corollary}
\newtheorem*{proposition}{Proposition}
\newtheorem*{lemma}{Lemma}

\newtheoremstyle{named}{}{}{}{}{\bfseries}{.}{.5em}{\thmnote{#3}}
\theoremstyle{named}
\newtheorem*{namedtheorem}{Theorem}

\newcounter{myalgctr}
\newenvironment{myalg}{%      define a custom environment
   \bigskip\noindent%         create a vertical offset to previous material
   \refstepcounter{myalgctr}% increment the environment's counter
   \textbf{Algorithm \themyalgctr}% or \textbf, \textit, ...
   \newline%
   }{\par\bigskip}  %
\numberwithin{myalgctr}{section}



\newenvironment{solution}{\begin{proof}[Solution]}{\end{proof}}


%%%%%%%%%%HYPERREFS PACKAGE%%%%%%%%%%%%%%%%%
\usepackage[colorlinks]{hyperref}
%\definecolor{webcolor}{rgb}{0.8,0,0.2}
%\definecolor{webbrown}{rgb}{.6,0,0}
%\usepackage[
%        colorlinks,
%       linkcolor=webbrown,  filecolor=webcolor,  citecolor=webbrown,
%        backref
%]{hyperref}
\usepackage[alphabetic, lite]{amsrefs} % for bibliography
\begin{document}
\thispagestyle{fancy}
\subsection*{Definitions}
\begin{namedtheorem}[First-order linear equation]A {\bf first-order linear differential equation} in the unknown $f(x)$ is a differential equation that can be written in the form
  \[
  f'(x)+p(x)f(x)=q(x) \tag{$*$}
  \]
Equation $(*)$ is called the {\bf standard form} of the equation.
\end{namedtheorem}
\begin{namedtheorem}[Integrating factor] Consider a first-order linear equation in the unknown $f(x)$ with standard form
  \[
  f'(x)+p(x)f(x)=q(x).
  \]
  An {\bf integrating factor} for this equation is any function of the form
  \[
  v(x)=e^{P(x)},
  \]
  where $P(x)$ is an antiderivative of $p(x)$. Using indefinite integral notation, we have
  \[
  v(x)=e^{\int p(x)\, dx}.
  \]
\end{namedtheorem}

%***********************************************


%***************************************

\subsection*{Procedures}
\begin{namedtheorem}[Solving first-order linear equations] Suppose $p, q$ are continuous on the interval $I$. To solve the differential equation with standard form
  \[
  f'(x)+p(x)f(x)=q(x), \ x\in I, \tag{$*$}
  \]
 proceed as follows:
 \begin{enumerate}
   \item Compute an antiderivative $P(x)$ of $p(x)$.
   \item Set $v(x)=e^{P(x)}$: i.e., $v(x)=e^{\int p(x)\, dx}$.
   \item The function $f(x)$ is a solution of $(*)$ if and only if it is a solution of
   \[
   (v(x)f(x))'=v(x)q(x).
   \]
   \item Find an antiderivative $G(x)$ of $v(x)q(x)$. Then the general solution of $(*)$ is
   \[
   f(x)=\frac{G(x)}{v(x)}+\frac{C}{v(x)},
   \]
   where $C$ is any constant. Using indefinite integral notation:
   \[
   f(x)=\frac{1}{v(x)}\int v(x)q(x)\, dx.
   \]
 \end{enumerate}
\end{namedtheorem}

%*********************************************************
\subsection*{Examples}
\begin{enumerate}
  \item Use the integrating factor method to find the general solution to $y'=k y$, where $k$ is any fixed constant.
  \item Consider the differential equation
  \[
  (x-2)f'=e^{-x}-3f, \ x\in (-\infty, 2).
  \]
  \begin{enumerate}
    \item Find the general solution to the differential equation.
    \item Find the solution satisfying $f(1)=-1$.
  \end{enumerate}
  \item Find the general solution to the differential equation
  \[
  (x^2+1)f'(x)-x=x^3-xf(x), \  x\in (-\infty, \infty).
  \]
\end{enumerate}




\end{document}
