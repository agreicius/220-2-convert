\documentclass[11pt]{article}
%%%%%%%%%%PACKAGES%%%%%%%%%%%%%%%%%%%%%%%%%%%%%%%%%%%
\usepackage{latexsym}
\usepackage{amssymb, amsmath, amsthm, amsfonts}
\usepackage{stmaryrd} %For \mapsfrom
%\usepackage[fleqn]{amsmath}  % fleqn option makes aligned equations flushed left!
%\usepackage[english]{babel}
%\usepackage{pgf}
\usepackage{mathtools}
\usepackage[mathscr]{eucal}
\usepackage{fancyhdr}
\usepackage{multicol,parcolumns}
\usepackage{enumerate}
%\usepackage{enumitem}
\usepackage[shortlabels]{enumitem}
\usepackage{graphicx}
\usepackage{extarrows}
\usepackage{cancel}
%\usepackage{tikz}
%\usepackage[all,cmtip]{xy} %\SelectTips{cm}{10}
\usepackage[all]{xy} \SelectTips{cm}{10}
%\usepackage{listings} %For code blocks

\input{LatexPreamble}
%%%%%%%%FANCY HEADER%%%%%%%%%
\pagestyle{plain}
\setlength{\headheight}{13.6pt}
\fancyhfoffset[L]{.5in}
%\lhead{\Large \bf{Name:}}
\chead{Antiderivatives}
\rhead{Math 240}
%\lfoot{TURN OVER!}
%\rfoot{TURN OVER!}

%%%%%%%PAGE LAYOUT%%%%%%%%%%%%%
\setlength{\textwidth}{6.5in}
\setlength{\textheight}{9in}

%\setlength{\topmargin}{-.8in}
%\setlength{\columnsep}{1.5in}
\addtolength{\hoffset}{-1 in}
\addtolength{\voffset}{-.5 in}


%%%%%%%THEOREM ENVIRONMENTS%%%%%%%%
\theoremstyle{definition}
\newtheorem*{definition}{Definition}
\newtheorem*{definitions}{Definitions}
\newtheorem*{notation}{Notation}
\newtheorem*{example}{Example}
\newtheorem*{comment}{Comment}
\newtheorem*{comments}{Comments}
\newtheorem*{examples}{Examples}
\newtheorem*{warning}{Warning}
\newtheorem*{theorem}{Theorem}
\newtheorem*{corollary}{Corollary}
\newtheorem*{proposition}{Proposition}
\newtheorem*{lemma}{Lemma}

\newtheoremstyle{named}{}{}{}{}{\bfseries}{.}{.5em}{\thmnote{#3}}
\theoremstyle{named}
\newtheorem*{namedtheorem}{Theorem}

\newcounter{myalgctr}
\newenvironment{myalg}{%      define a custom environment
   \bigskip\noindent%         create a vertical offset to previous material
   \refstepcounter{myalgctr}% increment the environment's counter
   \textbf{Algorithm \themyalgctr}% or \textbf, \textit, ...
   \newline%
   }{\par\bigskip}  %
\numberwithin{myalgctr}{section}



\newenvironment{solution}{\begin{proof}[Solution]}{\end{proof}}
\def\bpause{
\vspace{.1in}
\\
}
\def\pause{}
\newcommand{\alert}[1]{
\emph{#1}
}

%%%%%%%%%%HYPERREFS PACKAGE%%%%%%%%%%%%%%%%%
\usepackage[colorlinks]{hyperref}
%\definecolor{webcolor}{rgb}{0.8,0,0.2}
%\definecolor{webbrown}{rgb}{.6,0,0}
%\usepackage[
%        colorlinks,
%       linkcolor=webbrown,  filecolor=webcolor,  citecolor=webbrown,
%        backref
%]{hyperref}
\usepackage[alphabetic, lite]{amsrefs} % for bibliography
\begin{document}
\thispagestyle{fancy}
\subsection*{Definitions}
\begin{namedtheorem}[Antiderivative] Let $f$ be a real-valued function defined on an interval $I$. A function $F$ is called an antiderivative of $f$ if $F'(x)=f(x)$ for all $x\in I$.
\end{namedtheorem}

\begin{namedtheorem}[Indefinite integral] Let $f$ be a real-valued function defined on an interval $I$ and suppose $f$ has an antiderivative. The {\bf indefinite integral} of $f$ with respect to $x$ is the notation
  \[
  \int f \ dx
  \]
and is used to denote the general antiderivative of $f$. Thus if $F$ is a particular antiderivative, then we write
\[
\int f \ dx=F(x)+C
\]
to express the fact that the general antiderivative of $f$ is of the form $F(x)+C$ for some $C\in \mathbb{R}$. The symbol $\int$ is called the {\bf integral symbol}, the function $f$ is called the {\bf integrand} of the integral, and $x$ is called the {\bf variable of integration}.
\end{namedtheorem}
%********************************************
 \subsection*{Theory}
\begin{namedtheorem}[General antiderivative theorem] Let $f$ be a real-valued function defined on an interval $I$ and suppose $F$ is an antiderivative of $f$.
  \begin{enumerate}[itemsep=0pt, topsep=0pt]
    \item Given any $C\in \mathbb{R}$, the function $F(x)+C$ is an antiderivative of $f$.
    \item If $G$ is an antiderivative of $f$, then there is a $C\in \mathbb{R}$ such that
    \[
    G(x)=F(x)+C
    \]
    for all $x\in I$.
    \item Thus (1) and (2) imply that the {\bf general antiderivative} of $f$ on $I$ can be expressed as $F(x)+C$, where $C$ is any real number.
  \end{enumerate}

\end{namedtheorem}
\begin{namedtheorem}[Antiderivative formulas] The following antiderivative (or indefinite integral) formulas follow directly from a corresponding derivative formula.
  \begin{align*}
    \int 0\, dx&=C & \int x^r\, dx&=\frac{x^{r+1}}{r+1}+C, r\ne -1\\
    \int \cos kx\, dx&=\frac{1}{k}\sin kx+C & \int \sin kx\, dx &=-\frac{1}{k}\cos kx+C \\
    \int\sec^2kx\, dx&=\frac{1}{k}\tan kx+C & \int \csc^2 kx \, dx&=-\frac{1}{k}\cot kx+C\\
    \int \sec kx\tan kx\, dx&=\frac{1}{k}\sec kx+C & \int \csc x\cot x\, dx&=-\frac{1}{k}\csc kx+C
  \end{align*}
\end{namedtheorem}
\begin{namedtheorem}[Antiderivative rules] Let $f$ and $g$ be real-valued functions defined on an interval $I$. Suppose $F$ is an antiderivative of $f$ and $G$ is an antiderivative of $g$.
  \begin{enumerate}[itemsep=0pt, topsep=0pt]
    \item Given any constant $a\in \mathbb{R}$, the function $aF$ is an antiderivative of $af$, and hence
    \[
    \int af\, dx=aF(x)+C.
    \]
    \item The function $F(x)\pm G(x)$ is an antiderivative of $f(x)\pm g(x)$, and hence
    \[
    \int f\pm g\, dx=F(x)\pm G(x)+C.
    \]
  \end{enumerate}

\end{namedtheorem}
%*********************************************************
\subsection*{Examples}
\begin{enumerate}
  \item Find an antiderivative for the given function.
  \begin{enumerate}
    \item $\ds f(x)=x^7$
    \item $\ds f(x)=\frac{1}{\sqrt{x}}$
    \item $\ds f(x)=2\sin x-x^{2/3}$
  \end{enumerate}
  \item Find an antiderivative for the given function.
  \begin{enumerate}
    \item $f(x)=\sec^25x$
    \item $f(x)=2x\cos(x^2)$
    \item $f(x)=\cos(x^2)$
  \end{enumerate}
  \item At time $t=0$ minutes a tank containing 100 gallons of water begins leaking. After $t$ minutes the rate at which the water leaves the tank is given by
  \[
  r(t)=\frac{1}{\sqrt{2t+1}}.
  \]
  Let $f(t)$ be the amount of water in the tank after $t$ minutes. Find a formula for $f(t)$.

    \item Consider the differential equation
    \[
    f''(x)=-\frac{2}{3}\cos(2x)+x \tag{$*$}.
    \]
    \begin{enumerate}
      \item Find the general formula for a function $f(x)$ satisfying $(*)$.
      \item Find the unique function $f(x)$ satisfying $(*)$ and the initial conditions
      \[
      f(0)=0, f'(0)=-1.
      \]

  \end{enumerate}
\end{enumerate}




\end{document}
